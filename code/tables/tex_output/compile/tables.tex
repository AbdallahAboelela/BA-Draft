\documentclass[11pt, oneside]{article}   	
\usepackage[margin = 1 in]{geometry}                		
\geometry{letterpaper}                   		
%\usepackage[parfill]{parskip}    %Begin paragraphs with an empty line rather than an indent
\usepackage{graphicx}				
\usepackage{setspace}								
\usepackage{amssymb}
\doublespacing
%\geometry{footnotesep=2\baselineskip}
\interfootnotelinepenalty=10000 %prevents long footnotes from overflowing to new page
\usepackage{longtable}
\usepackage{caption} %Put caption* in a table to remove the enumeration
\usepackage{rotating} % To rotate the reviewer table
\usepackage{natbib}   %activates bibliography
\usepackage [english]{babel} %something for bibliography
\usepackage{verbatim} %to be able to use \begin{comment}
\usepackage{booktabs}
\usepackage{array}
\usepackage{float} 
\usepackage{pdflscape} %allows for one page to be landscape (lscape makes word landscape, pdflscape makes page landscape)
\usepackage{lscape}
\usepackage{threeparttable} %allows tablenotes
\usepackage[autolanguage]{numprint}
\usepackage{dirtytalk}

\begin{document}

% Table 1
\begin{table}[htbp]
\caption{Summary Statistics By Host Race: Listing Characteristics}
\begin{center}%
\small\begin{tabular}{l c | c | c c c c}
& \multicolumn{1}{c}{Full data} & \multicolumn{5}{c}{Regression Sample}
\\
 \cmidrule(r){3-7}
\\
 & \multicolumn{1}{c}{} & \multicolumn{1}{c}{All} & White & Black & Hispanic & Asian
\\
\hline\hline\noalign{\smallskip} 
 \textit{\textit{Outcome Variables}} & & & & & & \\ Price & 175.72 & 142.77 & 151.46 & 112.22 & 131.45 & 118.08 \\
 & (294.14) & (116.11) & (121.49) & (89.64) & (106.15) & (94.91) \\
 Number of Reviews & 17.51 & 16.87 & 17.46 & 15.39 & 16.75 & 14.23 \\
 & (31.86) & (31.23) & (32.37) & (27.52) & (29.92) & (26.77) \\
 \textit{Covariates} & & & & & & \\ \hline Property Type & & & & & & \\ \hspace{10bp}Apartments/Lofts    & 0.60 & 0.63 & 0.63 & 0.66 & 0.66 & 0.62 \\ \hspace{10bp}Townhouses/Condos   & 0.04 & 0.04 & 0.04 & 0.04 & 0.04 & 0.06 \\ \hspace{10bp}Houses                      & 0.32 & 0.30 & 0.30 & 0.27 & 0.26 & 0.30 \\ \hspace{10bp}Others                              & 0.04 & 0.03 & 0.03 & 0.03 & 0.03 & 0.03 \\Room Type &&&&&& \\ \hspace{10bp}Entire House/Apartment      & 0.58 & 0.55 & 0.58 & 0.43 & 0.51 & 0.42 \\ \hspace{10bp}Private Room                        & 0.38 & 0.42 & 0.39 & 0.50 & 0.44 & 0.53 \\ \hspace{10bp}Shared Room                         & 0.04 & 0.04 & 0.03 & 0.07 & 0.05 & 0.05 \\ Max Num. Guests & 3.44 & 3.15 & 3.24 & 2.94 & 3.06 & 2.84 \\
 & (2.41) & (2.13) & (2.15) & (2.06) & (2.15) & (2.00) \\
 Bedrooms & 1.34 & 1.26 & 1.28 & 1.19 & 1.21 & 1.18 \\
 & (0.92) & (0.80) & (0.83) & (0.69) & (0.78) & (0.72) \\
 Bathrooms & 1.30 & 1.23 & 1.24 & 1.19 & 1.21 & 1.19 \\
 & (0.69) & (0.55) & (0.56) & (0.49) & (0.52) & (0.53) \\
 Beds & 1.82 & 1.67 & 1.69 & 1.60 & 1.68 & 1.57 \\
 & (1.41) & (1.21) & (1.19) & (1.15) & (1.51) & (1.18) \\
 Cleaning Fee & 48.95 & 43.70 & 46.06 & 36.20 & 40.35 & 36.45 \\
 & (59.62) & (48.32) & (49.73) & (43.18) & (45.51) & (42.86) \\
 Extra Guests Charge & 13.74 & 13.43 & 13.26 & 15.13 & 13.94 & 12.72 \\
 & (23.65) & (22.67) & (23.00) & (22.71) & (22.48) & (20.36) \\
 Minimum Nights & 3.01 & 3.03 & 3.08 & 2.61 & 2.86 & 3.17 \\
 & (9.21) & (8.79) & (9.39) & (4.35) & (6.55) & (8.67) \\
 Availability (out of 30 days) & 11.54 & 11.04 & 10.64 & 14.19 & 11.24 & 10.79 \\
 & (10.93) & (10.91) & (10.75) & (11.49) & (10.94) & (11.01) \\
 Number of Amenities & 0.81 & 0.79 & 0.80 & 0.75 & 0.79 & 0.75 \\
 & (1.10) & (1.10) & (1.10) & (1.04) & (1.10) & (1.13) \\
 Instantly Bookable? & 0.15 & 0.15 & 0.14 & 0.21 & 0.17 & 0.16 \\
 & (0.36) & (0.36) & (0.34) & (0.41) & (0.38) & (0.37) \\
 Year of first review & 14.86 & 14.86 & 14.83 & 14.89 & 14.90 & 15.03 \\
 & (1.22) & (1.22) & (1.22) & (1.30) & (1.21) & (1.17) \\
 Strict Cancellation Policy & 0.43 & 0.40 & 0.40 & 0.41 & 0.41 & 0.40 \\\hline
Observations & \numprint{69007} & \numprint{45076} & \numprint{26391} & \numprint{3346} & \numprint{2274} & \numprint{3719}
\\
\hline\hline\noalign{\smallskip} \end{tabular} 
\begin{minipage}{6in}
{\it Note:} The values in the table are means and standard deviations of listing-level data in my full sample. Summary statistics for selected covariates are listed in the table. Categorical variables such as room type do not have standard deviations. Property types are explicitly listed if more than 1.5\% of listings are that type. Only the most popular cancellation policy type is listed - in the full sample, 99\% of listings have strict (43\%), flexible (31\%) or moderate (25\%) cancellation policies. Year of first review is a proxy for the time on the market - 14.86 indicates that the first review of the mean listing in the full sample occurred in October of 2014.
\end{minipage}
\end{center}
\end{table}

\newpage

% Table 2
\begin{table}[htbp]
\caption{Summary Statistics By Host Race: Host Demographics}
\begin{center}%
\small\begin{tabular}{l c | c | c c c c}
& \multicolumn{1}{c}{} & \multicolumn{5}{c}{Regression Sample}
\\
 \cmidrule(r){3-7}
\\
 & \multicolumn{1}{c}{Full data} & \multicolumn{1}{c}{All} & White & Black & Hispanic & Asian
\\
\hline\hline\noalign{\smallskip} 
 Race &&&&&& \\
 \hspace{10bp}White & 0.638 & 0.731 &  1.00 & 0.000 &  0.000 & 0.000 \\  \hspace{10bp}Black & 0.073 & 0.097 &  0.00 & 1.000 &  0.000 & 0.000 \\  \hspace{10bp}Hispanic & 0.051 & 0.065 &  0.00 & 0.000 &  1.000 & 0.000 \\  \hspace{10bp}Asian & 0.085 & 0.108 &  0.00 & 0.000 &  0.000 & 1.000 \\  Sex &&&&&& \\
 \hspace{10bp}Male & 0.310 & 0.448 &  0.451 & 0.403 &  0.498 & 0.436 \\  \hspace{10bp}Female & 0.378 & 0.552 &  0.549 & 0.597 &  0.502 & 0.564 \\  \hspace{10bp}Unknown & 0.000 & {0.000} & {0.000} &  {0.000}  & {0.000}  & {0.000} \\  Age &&&&&& \\
 \hspace{10bp}Young($<30$) & 0.427 & 0.513 &  0.494 & 0.542 &  0.518 & 0.612 \\  \hspace{10bp}Middle-aged & 0.421 & 0.464 &  0.478 & 0.449 &  0.471 & 0.378 \\  \hspace{10bp}Old ($>65$) &               0.019 & 0.021 &  0.026 & 0.005 &  0.010 & 0.008 \\ \hline
Observations & 69000 & 45076 & 32934 & 4354 & 2913 & 4875 
\\
\hline\hline\noalign{\smallskip} \end{tabular} 
\begin{minipage}{6in}
{\it Note:} The values in the table are summaries of host demographics inthe host-level data. Column 1 is the summary statistics for the full,unrestricted data set across 7 cities. Columns 2 $-$ 6 are the restricteddata used in the analysis. Column 2 is the full regression sample, andcolumns 3 $-$ 6 break down the regression sample by host race. The“Unknown” category was dropped from the regression and is therefore zerothroughout columns 2 $-$ 6. White refers only to Non-Hispanic Whites.\end{minipage}
\end{center}
\end{table}

\newpage

% Table 3
\begin{table}[htbp]
\caption{Summary Statistics By Host Race: Host Characteristics}
\begin{center}%
\small\begin{tabular}{l c | c | c c c c}
& \multicolumn{1}{c}{Full data} & \multicolumn{5}{c}{Regression Sample}
\\
 \cmidrule(r){3-7}
\\
 & \multicolumn{1}{c}{} & \multicolumn{1}{c}{All} & White & Black & Hispanic & Asian
\\
\hline\hline\noalign{\smallskip} 
 \textit{\textit{Outcome Variables}} & & & & & & \\ Host Listings Count & 5.53 & 2.23 & 2.16 & 2.38 & 2.49 & 2.44 \\
 & (32.99) & (2.59) & (2.50) & (2.83) & (3.03) & (2.61) \\
 \textit{Covariates} & & & & & & \\ \hline Review Scores Rating & 93.61 & 93.68 & 94.18 & 91.91 & 92.80 & 92.26 \\
 & (8.01) & (7.90) & (7.33) & (9.44) & (8.71) & (9.27) \\
 Host is a Superhost & 0.12 & 0.13 & 0.14 & 0.09 & 0.11 & 0.10 \\
 & (0.33) & (0.33) & (0.34) & (0.28) & (0.31) & (0.30) \\
 Response rate & 0.76 & 0.76 & 0.76 & 0.78 & 0.76 & 0.74 \\
 & (0.39) & (0.39) & (0.39) & (0.37) & (0.39) & (0.40) \\
 Acceptance rate & 0.45 & 0.45 & 0.46 & 0.35 & 0.49 & 0.44 \\
 & (0.46) & (0.46) & (0.46) & (0.45) & (0.47) & (0.47) \\
 Total good words in Summary & 0.68 & 0.69 & 0.69 & 0.70 & 0.70 & 0.63 \\
 & (0.89) & (0.89) & (0.89) & (0.92) & (0.91) & (0.85) \\
 Length of Summary & 242.42 & 243.37 & 243.69 & 239.80 & 239.06 & 247.00 \\
 & (110.80) & (110.47) & (108.22) & (111.69) & (108.49) & (124.61) \\
 Short words in Summary & 0.16 & 0.16 & 0.16 & 0.16 & 0.16 & 0.15 \\
 & (0.04) & (0.04) & (0.04) & (0.04) & (0.04) & (0.05) \\
 Response time $<$ 1 hour & 0.39 & 0.40 & 0.39 & 0.44 & 0.41 & 0.41 \\  &  &  &  &  &  &  \\ Host's Identity Verified? & 0.71 & 0.70 & 0.71 & 0.66 & 0.68 & 0.69 \\
 & (0.46) & (0.46) & (0.45) & (0.47) & (0.47) & (0.46) \\
 Guest Pic Required? & 0.04 & 0.04 & 0.04 & 0.06 & 0.04 & 0.04 \\
 & (0.19) & (0.19) & (0.19) & (0.23) & (0.19) & (0.19) \\
 Guest Phone Required? & 0.05 & 0.05 & 0.05 & 0.06 & 0.04 & 0.04 \\
 & (0.22) & (0.21) & (0.21) & (0.24) & (0.20) & (0.20) \\
\hline
Observations & \numprint{46930} & \numprint{45076} & \numprint{32934} & \numprint{4354} & \numprint{2913} & \numprint{4875}
\\
\hline\hline\noalign{\smallskip} \end{tabular} 
\begin{minipage}{6in}
{\it Note:} The values in the table are means and standard deviations of host-level data in the full sample. Summary statistics for selected covariates are listed in the table. Categorical variables such as response time do not have standard deviations. Statistics for only the most frequent response time (\say{within an hour}) are included. White refers only to non-Hispanic whites. Length of \say{Summary} and proportion of short words in the \say{Summary} refer to my constructed measures of host quality. These two measures were also calculated for the description, space, neighborhood overview, notes, and transit fields, but were not included in the table for the sake of clarity and because they follow a similar pattern as the table for the sake of clarity and because they follow a similar pattern as the \say{Summary} field.
\end{minipage}
\end{center}
\end{table}

\newpage

% Table 4
\begin{table}[htbp]
\caption{Summary Statistics By Race: Reviewer Characteristics}
\begin{center}%
\small\begin{tabular}{l c c c c c}
& \multicolumn{5}{c}{Regression Sample}
\\
 \cmidrule(r){2-6}
\\
 & \multicolumn{1}{c}{Full Sample} & White & Black & Hispanic & Asian
\\
\hline\hline\noalign{\smallskip} 
 Mean Sentiment per Review & 0.51 & 0.51 & 0.50 & 0.51 & 0.49 \\
 & (0.26) & (0.25) & (0.27) & (0.29) & (0.29) \\
 Mean Sentiment per Listing & 0.51 & 0.51 & 0.51 & 0.50 & 0.50 \\
 & (0.07) & (0.07) & (0.09) & (0.11) & (0.11) \\
\hline
Observations & 6912 & 5628 & 407 & 414 & 463
\\
\hline\hline\noalign{\smallskip} \end{tabular} 
\begin{minipage}{6in}
{\it Note:} The values in the table are means and standard deviations of reviewer-level data who left reviews for a randomly chosen set of hosts in Chicago. The review sentiment is the sentiment of each review, the listing sentiment is the average sentiment per listing. Observations in columns 2 - 5 do not add up to 17,050 because multiracial or unidentifiable reviewer pictures are excluded. White refers only to non-Hispanic whites.
\end{minipage}
\end{center}
\end{table}

\newpage

% Table 5
\begin{table}[htbp]
	\def\sym#1{\ifmmode^{#1}\else\(^{#1}\)\fi}
	\caption{Main result: Estimates of effect of host’s race and gender on price}
	\begin{tabular}{l*{5}{c}}
		\hline\hline
                    &\multicolumn{1}{c}{(1)}&\multicolumn{1}{c}{(2)}&\multicolumn{1}{c}{(3)}&\multicolumn{1}{c}{(4)}&\multicolumn{1}{c}{(5)}\\
                    &\multicolumn{1}{c}{Model 1}&\multicolumn{1}{c}{Model 2}&\multicolumn{1}{c}{Model 3}&\multicolumn{1}{c}{Model 4}&\multicolumn{1}{c}{Model 5}\\
\hline
White Male          &       31.25         &      -27.80         &       103.8         &       92.84         &       95.59         \\
                    &     (147.6)         &     (113.8)         &     (104.7)         &     (102.0)         &     (104.8)         \\
[1em]
White Female        &       26.23         &      -34.57         &       105.2         &       95.84         &       98.69         \\
                    &     (148.4)         &     (114.2)         &     (105.3)         &     (102.7)         &     (105.6)         \\
[1em]
Black Male          &      -28.02         &      -48.15         &       95.73         &       83.76         &       85.77         \\
                    &     (146.3)         &     (115.5)         &     (106.0)         &     (103.1)         &     (106.0)         \\
[1em]
Black Female        &      -19.77         &      -36.85         &       102.6         &       92.59         &       95.22         \\
                    &     (144.9)         &     (113.3)         &     (104.8)         &     (102.2)         &     (105.1)         \\
[1em]
Hispanic Male       &       9.268         &      -31.14         &       110.2         &       99.87         &       102.6         \\
                    &     (146.6)         &     (114.6)         &     (105.5)         &     (102.8)         &     (105.6)         \\
[1em]
Hispanic Female     &       12.35         &      -36.73         &       100.8         &       90.28         &       91.93         \\
                    &     (146.0)         &     (113.9)         &     (105.6)         &     (102.9)         &     (105.6)         \\
[1em]
Asian Male          &      -10.64         &      -48.36         &       92.58         &       80.60         &       82.58         \\
                    &     (144.5)         &     (113.1)         &     (104.3)         &     (101.6)         &     (104.4)         \\
[1em]
Asian Female        &      -20.84         &      -53.02         &       100.9         &       90.40         &       92.43         \\
                    &     (144.6)         &     (114.0)         &     (105.5)         &     (102.9)         &     (105.7)         \\
\hline
Location Fixed Effects&                     &         Yes         &         Yes         &         Yes         &         Yes         \\
Property Fixed Effects&                     &                     &         Yes         &         Yes         &         Yes         \\
Host Fixed Effects  &                     &                     &                     &         Yes         &         Yes         \\
\hline \vspace{-1.25em}&                     &                     &                     &                     &                     \\
Observations        &       46926         &       46926         &       46926         &       46902         &       46902         \\
Adjusted R2         &     0.00700         &       0.119         &       0.387         &       0.390         &       0.390         \\

	\hline\hline
	\multicolumn{5}{l}{\footnotesize Standard errors in parentheses}\\
	\multicolumn{5}{l}{\footnotesize \sym{*} \(p<0.05\), \sym{**} \(p<0.01\), \sym{***} \(p<0.001\)}\\
	\end{tabular}

	\begin{tablenotes}
	
	\item Notes: The dependent variable is the price of the listing. All race coefficients are relative to white males. The unit of observation is a listing. The sample is the sample of listings across 7 US cities. Model 1 is the baseline effect of host demographics on price. Model 2 controls for listing location to the neighborhood level. Model 3 adds listing characteristics such as property type, time on market, number of bedrooms, availability, etc. Model 4 adds host characteristics such as response and acceptance rates, measures of host effort, Superhost status, etc. See Data Appendix for full description of covariates.  
\end{tablenotes}
\end{table}

% Table 6
\begin{table}[htbp]\centering
	\def\sym#1{\ifmmode^{#1}\else\(^{#1}\)\fi}
	\caption{Robustness check with controls from Edelman \& Luca (2014), NYC data}
	\begin{tabular}{l*{1}{c}}
		\hline\hline
		                    &\multicolumn{1}{c}{(1)}\\
                    &\multicolumn{1}{c}{Edelman}\\
\hline
0                   &           0         \\
                    &         (.)         \\
[1em]
White               &      -2.787         \\
                    &     (70.53)         \\
[1em]
Black               &      -21.11         \\
                    &     (70.60)         \\
[1em]
Hispanic            &      -15.34         \\
                    &     (70.70)         \\
[1em]
Asian               &      -3.804         \\
                    &     (70.61)         \\
\hline
Location Fixed Effects&         Yes         \\
Property Fixed Effects&         Yes         \\
Host Fixed Effects  &         Yes         \\
\hline \vspace{-1.25em}&                     \\
Observations        &       12129         \\
Adjusted R2         &       0.224         \\

		\hline\hline
		\multicolumn{2}{l}{\footnotesize Standard errors in parentheses}\\
		\multicolumn{2}{l}{\footnotesize \sym{*} \(p<0.05\), \sym{**} \(p<0.01\), \sym{***} \(p<0.001\)}\\
	\end{tabular}
	\begin{tablenotes}
				
		\item Notes: This table presents the effect on price of controlling for Edelman \& Luca's (2014) full specification using my NYC data. My results are nearly identical to theirs (their coefficient on Black *hosts was -17.8) when controlling for similar covariates in the same city. The omitted category for race is White hosts. The omitted category for room type is Entire Apartment. I could not control for host social media accounts as a proxy for host reliability like Edelman \& Luca did, because Airbnb no longer provides this information. Instead, I controlled for ``host verified", a boolean for whether Airbnb has the host's phone number and email. I was not able to control for ``picture quality" either, but picture quality did not significantly influence price in Edelman \& Luca's regression.
	\end{tablenotes}
\end{table}

% Table 7
\begin{table}[htbp]\centering
	\def\sym#1{\ifmmode^{#1}\else\(^{#1}\)\fi}
	\caption{Estimates of effect of host’s race and gender on number of reviews}
	\begin{tabular}{l*{4}{c}}
		\hline\hline
		                    &\multicolumn{1}{c}{(1)}&\multicolumn{1}{c}{(2)}&\multicolumn{1}{c}{(3)}&\multicolumn{1}{c}{(4)}\\
                    &\multicolumn{1}{c}{Model 1}&\multicolumn{1}{c}{Model 2}&\multicolumn{1}{c}{Model 3}&\multicolumn{1}{c}{Model 4}\\
\hline
White Female        &      -0.804         &      -0.560         &      -1.420\sym{***}&      -1.310\sym{***}\\
                    &     (0.540)         &     (0.496)         &     (0.344)         &     (0.325)         \\
[1em]
Black Male          &      -2.049         &      -1.494         &      -2.034\sym{**} &      -1.146         \\
                    &     (1.137)         &     (0.970)         &     (0.617)         &     (0.592)         \\
[1em]
Black Female        &      -2.153         &      -1.439         &      -2.523\sym{***}&      -2.253\sym{***}\\
                    &     (1.125)         &     (1.004)         &     (0.559)         &     (0.536)         \\
[1em]
Hispanic Male       &      -1.404         &      -0.168         &      -0.183         &     0.00541         \\
                    &     (1.258)         &     (1.178)         &     (0.817)         &     (0.795)         \\
[1em]
Hispanic Female     &      -0.443         &       0.805         &      -0.867         &      -0.326         \\
                    &     (1.119)         &     (1.042)         &     (0.749)         &     (0.713)         \\
[1em]
Asian Male          &      -2.856\sym{**} &      -1.054         &      -1.024         &      -0.941         \\
                    &     (0.913)         &     (0.832)         &     (0.606)         &     (0.562)         \\
[1em]
Asian Female        &      -3.112\sym{**} &      -0.818         &      -1.201\sym{*}  &      -0.931         \\
                    &     (0.952)         &     (0.770)         &     (0.598)         &     (0.559)         \\
\hline
Location Fixed Effects&                     &         Yes         &         Yes         &         Yes         \\
Property Fixed Effects&                     &                     &         Yes         &         Yes         \\
Host Fixed Effects  &                     &                     &                     &         Yes         \\
\hline \vspace{-1.25em}&                     &                     &                     &                     \\
Observations        &       45072         &       45072         &       45072         &       45072         \\
Adjusted R2         &     0.00888         &      0.0630         &       0.408         &       0.455         \\

		\hline\hline
		\multicolumn{5}{l}{\footnotesize Standard errors in parentheses}\\
		\multicolumn{5}{l}{\footnotesize \sym{*} \(p<0.05\), \sym{**} \(p<0.01\), \sym{***} \(p<0.001\)}\\
	\end{tabular}
	\begin{tablenotes}

		\item Notes: The dependent variable is the number of reviews of the listing. The omitted category for race is white males, so all coefficients are relative to that group. The unit of observation is an Airbnb listing, so hosts who have multiple listings are treated separately each time. The sample is the sample of listings across 7 US cities. The specification is the same as Table 5. See Data Appendix for a discussion of my covariates.	
	\end{tablenotes}
\end{table}

% Table 8
\begin{table}[htbp]\centering
	\def\sym#1{\ifmmode^{#1}\else\(^{#1}\)\fi}
	\caption{Effect of host’s race on listing availability out of 30 days}
	\begin{tabular}{l*{1}{c}}
		\hline\hline
		                    &\multicolumn{1}{c}{(1)}\\
                    &\multicolumn{1}{c}{Number of vacant days out of 30}\\
\hline
White Female        &      -0.904\sym{***}\\
                    &     (0.123)         \\
[1em]
White Two people or Unknown&      -1.163\sym{***}\\
                    &     (0.149)         \\
[1em]
Black Male          &       2.113\sym{***}\\
                    &     (0.304)         \\
[1em]
Black Female        &       1.752\sym{***}\\
                    &     (0.272)         \\
[1em]
Black Two people or Unknown&       0.879         \\
                    &     (0.494)         \\
[1em]
Hispanic Male       &      -0.231         \\
                    &     (0.338)         \\
[1em]
Hispanic Female     &      -0.229         \\
                    &     (0.345)         \\
[1em]
Hispanic Two people or Unknown&      -0.857         \\
                    &     (0.519)         \\
[1em]
Asian Male          &      -0.240         \\
                    &     (0.302)         \\
[1em]
Asian Female        &      -1.210\sym{***}\\
                    &     (0.249)         \\
[1em]
Asian Two people or Unknown&      -0.483         \\
                    &     (0.417)         \\
[1em]
Multiracial or Unknown Male&       2.666\sym{***}\\
                    &     (0.682)         \\
[1em]
Multiracial or Unknown Female&       0.454         \\
                    &     (0.684)         \\
[1em]
Multiracial or Unknown Two people or Unknown&       1.221\sym{***}\\
                    &     (0.183)         \\
\hline
Location Fixed Effects&         Yes         \\
Property Fixed Effects&         Yes         \\
Host Fixed Effects  &         Yes         \\
\hline \vspace{-1.25em}&                     \\
Observations        &       68957         \\
Adjusted R2         &       0.224         \\

		\hline\hline
		\multicolumn{2}{l}{\footnotesize Standard errors in parentheses}\\
		\multicolumn{2}{l}{\footnotesize \sym{*} \(p<0.05\), \sym{**} \(p<0.01\), \sym{***} \(p<0.001\)}\\
	\end{tabular}
	
	\begin{tablenotes}
		
		\item This table presents the effect of host race on listing availability out of 30 days, controlling for my preferred specification in Table 5, Model 4. When a listing is booked, this availability metric is updated on the Airbnb website to reflect that booking. Therefore, this measure actually represents the number of days out of the total available days that listings were vacant, relative to a white male host.
	\end{tablenotes}
\end{table}

\newpage
% Table 9
	\begin{table}[htbp]\centering
		\def\sym#1{\ifmmode^{#1}\else\(^{#1}\)\fi}
		\caption{Robustness City}
		\begin{tabular}{l*{7}{c}}
			\hline\hline
			                    &\multicolumn{1}{c}{(1)}&\multicolumn{1}{c}{(2)}&\multicolumn{1}{c}{(3)}&\multicolumn{1}{c}{(4)}&\multicolumn{1}{c}{(5)}&\multicolumn{1}{c}{(6)}&\multicolumn{1}{c}{(7)}\\
                    &\multicolumn{1}{c}{LA}&\multicolumn{1}{c}{NYC}&\multicolumn{1}{c}{Austin}&\multicolumn{1}{c}{Chicago}&\multicolumn{1}{c}{New Orleans}&\multicolumn{1}{c}{DC}&\multicolumn{1}{c}{Nashville}\\
\hline
Black               &      -2.140         &      -30.59         &      -22.44         &      -5.119         &      -24.97\sym{*}  &      -15.82\sym{**} &      -16.53         \\
                    &     (8.483)         &     (27.29)         &     (20.55)         &     (5.020)         &     (9.445)         &     (5.621)         &     (10.24)         \\
[1em]
Hispanic            &       8.654         &      -34.40         &       12.48         &     -0.0855         &      -4.189         &      -2.363         &      -40.36\sym{**} \\
                    &     (8.023)         &     (27.66)         &     (11.39)         &     (5.714)         &     (12.29)         &     (7.185)         &     (13.55)         \\
[1em]
Asian               &       0.939         &      -29.50         &      -45.73\sym{***}&      -16.85\sym{***}&      -3.150         &      -14.33\sym{*}  &      -5.671         \\
                    &     (5.852)         &     (27.30)         &     (12.07)         &     (4.175)         &     (16.86)         &     (6.042)         &     (23.82)         \\
[1em]
Multiracial or Unknown&       7.168         &      -29.50         &       2.283         &       13.52         &      -13.03         &       4.361         &      -43.06\sym{**} \\
                    &     (6.057)         &     (26.99)         &     (15.28)         &     (7.754)         &     (7.951)         &     (6.297)         &     (12.67)         \\
[1em]
Los Angeles         &           0         &                     &                     &                     &                     &                     &                     \\
                    &         (.)         &                     &                     &                     &                     &                     &                     \\
[1em]
0                   &                     &           0         &                     &                     &                     &                     &                     \\
                    &                     &         (.)         &                     &                     &                     &                     &                     \\
[1em]
New York City       &                     &           0         &                     &                     &                     &                     &                     \\
                    &                     &         (.)         &                     &                     &                     &                     &                     \\
[1em]
Austin              &                     &                     &           0         &                     &                     &                     &                     \\
                    &                     &                     &         (.)         &                     &                     &                     &                     \\
[1em]
Chicago             &                     &                     &                     &           0         &                     &                     &                     \\
                    &                     &                     &                     &         (.)         &                     &                     &                     \\
[1em]
New Orleans         &                     &                     &                     &                     &           0         &                     &                     \\
                    &                     &                     &                     &                     &         (.)         &                     &                     \\
[1em]
Washington DC       &                     &                     &                     &                     &                     &           0         &                     \\
                    &                     &                     &                     &                     &                     &         (.)         &                     \\
[1em]
Nashville           &                     &                     &                     &                     &                     &                     &           0         \\
                    &                     &                     &                     &                     &                     &                     &         (.)         \\
\hline
Location Fixed Effects&         Yes         &         Yes         &         Yes         &         Yes         &         Yes         &         Yes         &         Yes         \\
Property Fixed Effects&         Yes         &         Yes         &         Yes         &         Yes         &         Yes         &         Yes         &         Yes         \\
Host Fixed Effects  &         Yes         &         Yes         &         Yes         &         Yes         &         Yes         &         Yes         &         Yes         \\
\hline \vspace{-1.25em}&                     &                     &                     &                     &                     &                     &                     \\
Observations        &       26076         &       20417         &        5818         &        5144         &        4511         &        3722         &        3277         \\
Adjusted R2         &       0.438         &       0.280         &       0.468         &       0.366         &       0.475         &       0.450         &       0.616         \\

			\hline\hline
			\multicolumn{8}{l}{\footnotesize Standard errors in parentheses}\\
			\multicolumn{8}{l}{\footnotesize \sym{*} \(p<0.05\), \sym{**} \(p<0.01\), \sym{***} \(p<0.001\)}\\
		\end{tabular}
		\begin{tablenotes}

			\item Notes: The effects for the combined data from Table 3 are now broken down across all 7 cities. The cities decrease in number of observations from left to right. Each set of coefficients represents the coefficient on host race, with price as the outcome variable. I control for my preferred specification that includes listing location, listing characteristics, and host characteristics. Low number of observations for Black, Hispanic, and Asian hosts contribute to imprecise estimates in smaller cities. New Orleans and Nashville have less than 100 Hispanic and Asian hosts. DC and Austin have less than 200 Hispanic and Asian hosts. 
		\end{tablenotes}
	\end{table}


\newpage
\begin{comment}
% Table 10
%%%%Careful; When we rerun the stata do file, the "\$" is read as $
\newpage
\begin{landscape}
	\begin{table}[htbp]\centering
		\def\sym#1{\ifmmode^{#1}\else\(^{#1}\)\fi}
		\caption{Robustness Listing Characteristics}
		\begin{tabular}{l*{9}{c}}
			\hline\hline
			                    &\multicolumn{1}{c}{(1)}&\multicolumn{1}{c}{(2)}&\multicolumn{1}{c}{(3)}&\multicolumn{1}{c}{(4)}&\multicolumn{1}{c}{(5)}&\multicolumn{1}{c}{(6)}&\multicolumn{1}{c}{(7)}&\multicolumn{1}{c}{(8)}&\multicolumn{1}{c}{(9)}\\
                    &\multicolumn{1}{c}{Low $ LA}&\multicolumn{1}{c}{High $ LA}&\multicolumn{1}{c}{Low $ NY}&\multicolumn{1}{c}{High $ NY}&\multicolumn{1}{c}{Older Listings}&\multicolumn{1}{c}{Newer Listings}&\multicolumn{1}{c}{Apartments}&\multicolumn{1}{c}{Condos}&\multicolumn{1}{c}{Houses}\\
\hline
Black               &       8.484         &      -20.53         &       0.943         &       3.865         &      -1.747         &      -8.164\sym{***}&      -2.718         &      -7.304         &      -13.45         \\
                    &     (11.03)         &     (11.78)         &     (1.860)         &     (10.13)         &     (6.942)         &     (1.820)         &     (2.648)         &     (11.16)         &     (11.14)         \\
[1em]
Hispanic            &       3.847         &      -13.61         &      -4.802         &      -6.285         &       3.999         &       2.589         &      -3.703         &      -18.38         &       9.505         \\
                    &     (6.405)         &     (18.78)         &     (3.139)         &     (5.966)         &     (7.187)         &     (3.319)         &     (1.887)         &     (9.518)         &     (10.53)         \\
[1em]
Asian               &      -5.738\sym{**} &      -4.347         &       5.217         &      -1.648         &      -9.855\sym{***}&      -1.425         &      -2.103         &      -21.21\sym{*}  &      -10.04         \\
                    &     (1.949)         &     (16.21)         &     (8.339)         &     (8.934)         &     (2.908)         &     (3.453)         &     (3.810)         &     (10.57)         &     (6.042)         \\
[1em]
Multiracial or Unknown&      -0.227         &       3.819         &      -1.555         &       4.168         &      -2.160         &      -3.304         &      -0.414         &      -11.94         &       16.39         \\
                    &     (1.391)         &     (14.43)         &     (2.210)         &     (10.40)         &     (4.439)         &     (3.243)         &     (2.541)         &     (15.56)         &     (8.691)         \\
\hline
Location Fixed Effects&         Yes         &         Yes         &         Yes         &         Yes         &         Yes         &         Yes         &         Yes         &         Yes         &         Yes         \\
Property Fixed Effects&         Yes         &         Yes         &         Yes         &         Yes         &         Yes         &         Yes         &         Yes         &         Yes         &         Yes         \\
Host Fixed Effects  &         Yes         &         Yes         &         Yes         &         Yes         &         Yes         &         Yes         &         Yes         &         Yes         &         Yes         \\
\hline \vspace{-1.25em}&                     &                     &                     &                     &                     &                     &                     &                     &                     \\
Observations        &       17155         &        8921         &       10717         &        9700         &       15193         &       39268         &       41254         &        2903         &       22236         \\
Adjusted R2         &      0.0968         &       0.480         &      0.0513         &       0.326         &       0.547         &       0.477         &       0.312         &       0.416         &       0.461         \\

			\hline\hline
			\multicolumn{10}{l}{\footnotesize Standard errors in parentheses}\\
			\multicolumn{10}{l}{\footnotesize \sym{*} \(p<0.05\), \sym{**} \(p<0.01\), \sym{***} \(p<0.001\)}\\
		\end{tabular}
		\begin{tablenotes}
			\item \footnotesize Standard errors in parentheses
			\item \footnotesize \sym{*} \(p<0.05\), \sym{**} \(p<0.01\), \sym{***} \(p<0.001\)
			\item Notes: I break down my combined data by price, time on market, and property type. The categories, from left to right, are: listings whose price is below vs. above the mean price of \$147 and whose prices are above \$800, the price originally dropped, listings who have have been on the market for no more than 2 years vs. no more than 8 years, and listings of different property types, including apartments (includes apartments and lofts), condos (includes condos and townhouse), and ho uses. I control for my preferred specification throughout. The outcome variable is price of the listing.
		\end{tablenotes}
	\end{table}
\end{landscape}
%%%%Careful; When we rerun the stata do file, the "\$" is read as $
\end{comment}

\centering
% Table 11
\begin{landscape}
\begin{table}[htbp]\centering
	\def\sym#1{\ifmmode^{#1}\else\(^{#1}\)\fi}
	\caption{Estimates of effect of host demographics on review sentiment, by reviewer demographics}
	\begin{tabular}{l *{8}{c}}
		\hline\hline
		&\multicolumn{8}{c}{Reviewers} \\
		\cmidrule(r){2-9}\\
		                    &\multicolumn{1}{c}{(1)}&\multicolumn{1}{c}{(2)}&\multicolumn{1}{c}{(3)}&\multicolumn{1}{c}{(4)}&\multicolumn{1}{c}{(5)}&\multicolumn{1}{c}{(6)}&\multicolumn{1}{c}{(7)}&\multicolumn{1}{c}{(8)}\\
                    &\multicolumn{1}{c}{White M}&\multicolumn{1}{c}{White F}&\multicolumn{1}{c}{Black M}&\multicolumn{1}{c}{Black F}&\multicolumn{1}{c}{Hispanic M}&\multicolumn{1}{c}{Hispanic F}&\multicolumn{1}{c}{Asian M}&\multicolumn{1}{c}{Asian F}\\
\hline
White Female        &     -0.0792         &      0.0339         &       0.932\sym{*}  &      -0.328         &       0.159         &      0.0118         &      0.0759         &      0.0504         \\
                    &    (0.0696)         &    (0.0492)         &     (0.443)         &     (0.277)         &     (0.331)         &     (0.204)         &     (0.124)         &     (0.112)         \\
[1em]
Black Male          &      -0.183         &      -0.164         &       2.069         &      -0.173         &       1.060         &      -3.739\sym{**} &       0.114         &      -1.394\sym{***}\\
                    &     (0.206)         &     (0.296)         &     (1.347)         &     (1.058)         &     (0.856)         &     (1.148)         &     (0.350)         &     (0.258)         \\
[1em]
Black Female        &     -0.0918         &      0.0250         &      -2.707\sym{**} &      0.0774         &       0.388         &      0.0282         &       1.068\sym{***}&      -0.364         \\
                    &     (0.174)         &     (0.110)         &     (0.899)         &     (0.669)         &     (0.555)         &     (0.690)         &     (0.239)         &     (0.222)         \\
[1em]
Hispanic Male       &     -0.0475         &      0.0668         &      -0.354         &       0.808         &      -0.500         &      -0.473         &      -0.101         &     0.00660         \\
                    &     (0.102)         &     (0.132)         &     (0.881)         &     (0.622)         &     (0.584)         &     (0.869)         &     (0.205)         &     (0.316)         \\
[1em]
Hispanic Female     &      0.0109         &     -0.0975         &       4.097\sym{*}  &      -1.371         &      -1.062         &       1.374\sym{*}  &      0.0664         &      -0.225         \\
                    &     (0.361)         &    (0.0811)         &     (1.698)         &     (2.849)         &     (0.860)         &     (0.520)         &     (0.206)         &     (0.484)         \\
[1em]
Asian Male          &      -0.340         &      -0.257         &       3.599\sym{***}&       0.307         &     -0.0628         &      -1.334         &       0.679\sym{*}  &       0.165         \\
                    &     (0.243)         &     (0.174)         &     (0.535)         &     (0.804)         &     (1.075)         &     (1.822)         &     (0.258)         &     (0.279)         \\
[1em]
Asian Female        &      -0.285         &      -0.290\sym{*}  &       7.117\sym{***}&       0.991         &      -0.757         &      -0.691         &     -0.0732         &      -0.588         \\
                    &     (0.167)         &     (0.139)         &     (1.668)         &     (0.612)         &     (0.960)         &     (0.527)         &     (0.228)         &     (0.322)         \\
\hline
Location Fixed Effects&         Yes         &         Yes         &         Yes         &         Yes         &         Yes         &         Yes         &         Yes         &         Yes         \\
Property Fixed Effects&         Yes         &         Yes         &         Yes         &         Yes         &         Yes         &         Yes         &         Yes         &         Yes         \\
Host Fixed Effects  &         Yes         &         Yes         &         Yes         &         Yes         &         Yes         &         Yes         &         Yes         &         Yes         \\
\hline \vspace{-1.25em}&                     &                     &                     &                     &                     &                     &                     &                     \\
Observations        &        2690         &        2557         &         124         &         171         &         201         &         145         &         487         &         537         \\
Adjusted R2         &      0.0495         &      0.0452         &       0.803         &       0.622         &       0.499         &       0.684         &       0.234         &       0.193         \\
	
		\hline\hline
		\multicolumn{9}{l}{\footnotesize Standard errors in parentheses}\\
		\multicolumn{9}{l}{\footnotesize \sym{*} \(p<0.05\), \sym{**} \(p<0.01\), \sym{***} \(p<0.001\)}\\
	\end{tabular}
\end{table}
\end{landscape}

% Table 12
\begin{table}[htbp]\centering
	\def\sym#1{\ifmmode^{#1}\else\(^{#1}\)\fi}
	\caption{Estimates of effect of host's race and gender on yearly revenue}
	\begin{tabular}{l*{4}{c}}
		\hline\hline
		                    &\multicolumn{1}{c}{(1)}&\multicolumn{1}{c}{(2)}&\multicolumn{1}{c}{(3)}&\multicolumn{1}{c}{(4)}\\
                    &\multicolumn{1}{c}{Model 1}&\multicolumn{1}{c}{Model 2}&\multicolumn{1}{c}{Model 3}&\multicolumn{1}{c}{Model 4}\\
\hline
White Female        &      -199.0\sym{***}&      -156.4\sym{***}&      -151.9\sym{***}&      -144.3\sym{***}\\
                    &     (48.54)         &     (46.80)         &     (39.72)         &     (40.12)         \\
[1em]
Black Male          &      -655.0\sym{***}&      -329.5\sym{***}&      -261.8\sym{***}&      -168.7\sym{**} \\
                    &     (98.27)         &     (96.16)         &     (59.77)         &     (59.90)         \\
[1em]
Black Female        &      -814.7\sym{***}&      -365.0\sym{***}&      -319.5\sym{***}&      -290.0\sym{***}\\
                    &     (96.68)         &     (78.69)         &     (51.94)         &     (46.56)         \\
[1em]
Hispanic Male       &      -209.0         &      -44.58         &      -25.42         &      -2.148         \\
                    &     (112.5)         &     (97.48)         &     (88.24)         &     (84.18)         \\
[1em]
Hispanic Female     &      -280.8\sym{*}  &      -79.78         &      -119.0         &      -57.80         \\
                    &     (140.1)         &     (120.7)         &     (108.2)         &     (106.6)         \\
[1em]
Asian Male          &      -360.5\sym{**} &      -95.77         &      -15.85         &      -16.36         \\
                    &     (129.4)         &     (115.4)         &     (88.42)         &     (84.27)         \\
[1em]
Asian Female        &      -676.6\sym{***}&      -329.2\sym{***}&      -183.6\sym{**} &      -162.6\sym{**} \\
                    &     (98.12)         &     (74.93)         &     (62.31)         &     (60.16)         \\
\hline
Location Fixed Effects&                     &         Yes         &         Yes         &         Yes         \\
Property Fixed Effects&                     &                     &         Yes         &         Yes         \\
Host Fixed Effects  &                     &                     &                     &         Yes         \\
\hline \vspace{-1.25em}&                     &                     &                     &                     \\
Observations        &       45072         &       45072         &       45072         &       45072         \\
Adjusted R2         &     0.00628         &      0.0959         &       0.361         &       0.413         \\

		\hline\hline
		\multicolumn{5}{l}{\footnotesize Standard errors in parentheses}\\
		\multicolumn{5}{l}{\footnotesize \sym{*} \(p<0.05\), \sym{**} \(p<0.01\), \sym{***} \(p<0.001\)}\\
	\end{tabular}

	\begin{tablenotes}

		\item Notes: The dependent variable is a constructed measure of yearly host revenue, as measured by (price * number of reviews per month * 12) for each listing. The omitted category for race is white males, so all coefficients are relative to that group. The unit of observation is an Airbnb listing, so hosts who have multiple listings are treated separately each time. The sample is the sample of listings across 7 US cities. The specification is the same as Table 5. See Data Appendix for a full discussion of my covariates.
	\end{tablenotes}
\end{table}

\end{document}



