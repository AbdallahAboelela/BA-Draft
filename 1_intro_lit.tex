African-Americans experience pervasively worse outcomes in the housing market as a result of historic and current racial discrimination \citep{krysan}. Even after the gains during the Civil Rights Era, such as the landmark Fair Housing Act of 1968, discrimination in the housing market is widely documented by social scientists. African-American renters are told that there are 30\% fewer available housing units than white renters \citep{yinger1}. African-American families face higher barriers when raising capital to purchase a home \citep{pope}. E-mails sent to landlords from home-seekers with typical African-American names receive lower response rates than emails sent by those with names commonly associated with whites \citep{hanson}.

%\citep{yinger1}
%Landlords renting out apartments discriminate both because of their own prejudice and in response to the prejudice of their white renters \citep{ondrich}.

Economists have primarily studied discrimination against African-American tenants. There is little research on the other side of the market - when African-Americans are supplying, rather than demanding, housing. Because property ownership cannot be randomized, it is difficult to disentangle true discrimination from systematic differences in the housing owned by African-Americans and white landlords.%
	\footnote{One would also expect black landlords to fare worse than white landlords in this area as well. Properties owned by African-Americans tend to be less expensive than those owned by white Americans. The average black household still has less mean wealth than a white household \citep{oliver}. Even middle-class black and Hispanic households still live in neighborhoods with median incomes similar to those of very poor white neighborhoods \citep{reardon}.} 
Some studies have found evidence that African-American homeowners are more likely to be targeted by subprime loans \citep{foreclosure} or pay more than whites for similar housing \citep{bayer, myers}, but few studies have a credible identification strategy to separate discrimination from correlated observables. 

%Some studies have found correlatory evidence of discrimination against African-American homeowners, but none have a credible mechanism to identified discrimination from other factors. For instance, the spike in subprime lending and the ensuing foreclosure crisis was found to be causally linked to residential segregation, and evidence suggests that specifically black residential dissimilarity and spatial isolation were important predictors of foreclosures across U.S. metropolitan areas \citep{foreclosure}.

This paper leverages the rise of Peer-to-Peer (P2P) marketplaces sharing economies and the data and standardization their platforms provide to credibly measure discrimination against minority landlords. Airbnb is a sharing economy platform that allows people to rent out their apartment, house, or a single room to short-term lodgers. Since Airbnb simply provides an online platform for a market that already exists, it is reasonable to assume that agents will discriminate on Airbnb in a similar way that they discriminate in the real world. This means that studying discrimination in sharing economies could be an important way to learn about discrimination in traditional housing markets as well.
%Sharing economies are convenient for research in several distinct ways. First, all agents on the platform obey a uniform set of rules and observe the same set of information about one another. Second, sharing economies allow for cleaner data collection, as well as both price and a proxy for quantity demand for each listing. 

Measurements of discrimination on Airbnb are still potentially confounded by many other factors that affect a listing's price. \cite{edelman}, \cite{wang}, and \cite{kakar} have conducted small-sample studies of discrimination against Airbnb hosts in single-city contexts. \cite{edelman}, the first study of this kind, estimated the effect of host race on the price of their listing in New York City with a small set of controls. Their results suggest that non-black hosts on Airbnb have prices roughly 12\% higher than black hosts. However, they only control for a few property characteristics, the quality of the host's reviews, and a measure of the reliability of the host. They leave out many other unobservables such as the type of listing (which is important if black hosts own single rooms and white hosts own entire houses) and proxies for the quality of the host themselves (which could also vary by race). 

In this paper, I use a dataset from a webscrape of Airbnb to empirically measure discrimination. Using price and quantity information for 70,000 Airbnb hosts throughout the country, I address the limitations of Edelman and Luca's research by controlling for location, additional property characteristics, comprehensive measures of listing size, and text analyses of host-written descriptions of the listing. 

I find that non-white hosts, both male and female, have lower prices than white hosts. The biggest effect is for Asian female hosts, whose prices are roughly \$9 (6\%) less per day than white male hosts who own the same type of listing. The second biggest effect is for black males, whose price is lower by \$7 (4.5\%), followed by black women and Asian men with price disparities of \$6 (4\%) per day, and Hispanic females with a coefficient of \$5 (3.3\%).%
	\footnote{This effect is statistically significant at the p $<$ .001 level for black hosts and Asian females, the p $<$ .01 level for Asian male hosts, and p $<$ .05 level for Hispanic females.} 
For Hispanic men the effect is small, around \$2 (1.3\%), and is not statistically significant. 

Crucially, I also have data on the number of reviews and the number of vacancies of a listing, two measures of the quantity demanded. This is important because knowing both quantity demanded and price allows me to distinguish whether the price disparity between minority and white hosts on Airbnb is due to a demand shift (consistent with discrimination) or a supply shift (consistent with a difference in marginal cost for the hosts, or other hypotheses). The presence of discrimination would mean that despite lower prices, minority hosts face lower quantity demanded. 

My first measure of quantity demanded is the number of reviews. I find that black hosts and white female hosts have 1 - 2 fewer reviews than white hosts for a listing that spent the same amount of time on the market. Effects for all other hosts are slightly negative, but not significant.%
	\footnote{This conclusion is only salient if the total number of reviews is a reasonable proxy for the demand of a listing. Yet, one can imagine that if reviewers systematically under-review minority hosts relative to white hosts, these groups would have lower numbers of reviews that do not necessarily represent a lower quantity demanded. There is no way to tell apart these mechanisms in my data. A recent study found that reviews left by hosts on guests’ pages can significantly reduce discrimination and render acceptance rates of guests with white-sounding names and African American-sounding names statistically indistinguishable \citep{cui}, but it remains unknown whether or not reviewers discriminate against minorities in leaving reviews \citep{ye}. If reviewers systematically under-review minority hosts, this itself could be evidence of discrimination. My working assumption is that even if not every guest leaves a review, the review proportion is similar across host race, and a lower number of reviews therefore indicates a real difference between quantity demanded of minority hosts and white hosts.}

One potential explanation for a lower number of reviews is that a minority host might make their listing available to guests less frequently than a white host. A host controls how many days of the month they offer their listing for rent via an availability calendar on the listing page. When a guest books their listing, the booked days disappear from the availability calendar. Therefore, the measure of availability is actually a measure of true vacancy for the listing. If a black host has a lower number of reviews, perhaps this is because they offer their listing for fewer days of the month than white hosts. To test this, I regress a listing's availability out of 30 days on host race. Results show that contrary to this hypothesis, the listings of black hosts stay vacant 2 - 3 days per month \textit{longer} than the listings of white hosts. The listings of white female hosts and Asian female hosts, on the other hand, are vacant less frequently than white hosts.

In sum, no minority host group has a higher number of reviews than white hosts. For those that have a lower number of reviews, the reasons differ. Black hosts fare the worst - both measures of quantity demanded confirm that despite lower prices, the listings of black hosts have fewer guests, and stay vacant longer than the listings of white hosts. This is consistent with the presence of discrimination for black hosts. However, at least part of the lower number of reviews for Asian females and white females can be explained by differences in availability. Therefore, the overall effect on quantity demanded for these groups is ambiguous. There are no effects for Hispanic hosts. 

I conduct several robustness checks of the main result. First, I find that prices are lower for black hosts across all cities in my sample. Prices for Hispanic and Asian hosts are lower as well, with several exceptions in cities with a small sample size of Hispanic and Asian hosts, such as Nashville and DC. This indicates that the price disparity is not driven by a single city. Second, I break up my sample by various listing characteristics (price, type of property, and location), and find that the price disparity holds across all types of listings. 

Finally, I investigate whether or not the price disparity is driven by minority hosts owning listings of worse quality, or simply being worse hosts, than whites. I consider the quality of a host's reviews as a proxy for the quality of the listing and host. I use the race and gender of the reviewer and the host to compare the sentiment (how favorable or unfavorable the review is) of the reviews that guests leave for white and for minority hosts.%
	\footnote{Since it required hand-coding, demographic information of the reviewers is only available for a randomly-chosen subset of hosts in Chicago.} 
Rather than observing that minority hosts uniformly had lower quality reviews, which the hypothesis would predict, the significance of the result was either negligible, or depended on the demographics of the reviewer and host. While there is some evidence that male reviewers tend to rate male hosts higher, there is little within-race preference between reviewers and hosts. Taken as a whole, sentiment analysis suggests that minority hosts do not have lower quality reviews. 






\subsection{About Airbnb} 
\label{about}
Airbnb is an online marketplace founded in 2008 that allows hosts to rent their private dwellings to guests as temporary accommodation. As of 2017, it has more than 3 million listings, more than Marriott's 1.2 million rooms worldwide \citep{aboutus}. Just like traditional hotel chains, guests on Airbnb can browse listings by city and property type, and book a stay based on prices, location, past reviews, pictures of the listing, size, and amenities. Unlike traditional hotel chains, however, hosts create a profile for themselves and a page for each listing they are renting. Each listing page includes the name and picture of the host, the reviews left by previous guests, and those guests' profile pictures. Guests can therefore infer demographic information about the host through a host's picture and name, creating the opportunity for discrimination. Figures \ref{fig:listing}-\ref{fig:property} present screenshots of a listing in a Chicago neighborhood, illustrating some of the information that would be available to a potential guest. Figures \ref{fig:reviewinfo} - \ref{fig:location} display an example of reviews, a sample host profile, and sample location maps, and can be found in the Appendix.










\subsection{Previous Literature} 
\label{previous}

\textbf{Theoretical foundation}

\cite{becker} proposed the idea that discrimination against a group is reflected in the prices that that group charges in a particular market, be it labor or products. In the Airbnb market, Becker's market discrimination would be reflected in the price that the guest (buyer) pays to the host (seller) to stay with them. If the guest is discriminating, then given two comparable listings, they would choose not to stay in the one owned by a minority host. Responding to (or anticipating) a lower demand, hosts in minority groups rationally post a lower price and, despite this, face a lower quantity demanded. 

This type of market discrimination laid out by Becker rests on the idea that groups who are discriminated against face lower demand in the market, which drives down their prices. But ex ante, it is unclear if a hypothetical price disparity measured between minority and white hosts is a result of a supply or a demand shift. There could be multiple explanations for a lower supply curve rather than demand curve. For instance, minority hosts could charge a lower price for their listing because it is cheaper for them, on the margin, to operate a similar listing relative to a white host. Since Black and Hispanic workers tend to earn less than their White counterparts, even for the same amount of education, they may have a lower opportunity cost of time \citep{wages}. Minority hosts would therefore have a lower marginal cost of managing their listing, and so would choose to set lower prices than White hosts with comparable listings. If the price disparity was due to these marginal cost differences, then the quantity demanded of minority hosts' listings should be higher than those of White hosts. 

Basic microeconomic theory says that we can test this by examining the quantity demanded. If prices are lower because the supply curve is lower, then minority hosts would have a higher quantity demanded. Conversely, if the prices are lower because the demand curve is lower - which would be in line with the presence of discrimination - then the quantity demanded should be lower than it is for white hosts. 

Becker was concerned with discrimination arising from face-to-face interactions between minority and majority groups. Since then, there has been a large amount of research indicating that Becker's theory holds for people participating in online markets for labor, lending, rental, and products. In these cases, participants simply bring their prejudices online and use names and photos to discriminate. Next, I detail the research that explores how the theory of discrimination plays out in online markets. 

\vspace{5mm}
\textbf{Research on discrimination in P2P online commerce}

As commerce has moved online, many transactions now occur on peer-to-peer (P2P) and sharing economy platforms such as Etsy, TaskRabbit, and Airbnb. Unlike traditional markets, participants on these platforms most often do not meet one another in person before agreeing to a transaction. Moreover, individuals are not likely to have the reputational safety of a brick-and-mortar store, and thus have a harder time credibly guaranteeing the quality of their product. These factors contribute to a higher perceived risk of transacting online. 

As a result, many platforms have instituted measures to mitigate this risk. Most have user profiles and encourage users to post their names and photos, as well as descriptions of themselves and their products to bolster credibility. Platforms often solicit reviews from previous buyers that are posted alongside the product, or on the seller's profile, to aid future buyers by increasing transparency and thereby encouraging transactions. 

However, the presence of identifiable information about a person’s demographics creates the opportunity to use user information to discriminate in [similar ways as one would in person]. Indeed, a growing body of research indicates that P2P market participants do exactly that. \cite{doleac} examined the effect of apparent race on market outcomes when selling an iPod on various online marketplaces. In some pictures, a dark-skinned hand was holding the iPod, signaling a Black seller, while in others, a light-skinned hand was holding the iPod, signaling a White seller. Hands which indicate a Black seller received 18\% fewer and 11\% lower offers than White sellers. Furthermore, bidders were less likely to include their name in offers made to Black sellers. \cite{pope} found that in a P2P lending market (Prosper.com) demographic characteristics conveyed through pictures and text significantly affected loan terms for Black borrowers. Black borrowers were 25\% to 35\% less likely to receive loans than White borrowers with similar credit profiles, and loans received by Black borrowers had an interest rate 60 to 80 basis points higher than White borrowers.

In sharing economies, a similar pattern occurs. \cite{knittel} explored the effect of race on market outcomes in rideshare platforms. These platforms provide rider information such as the first name, photo, and rating to drivers before (Lyft) or at (Uber) the time of ride acceptance. The authors ordered 1,500 Uber and Lyft trips to measure the impact of rider race on fare and wait times, varying the apparent race of the rider in these photos, as well as degree to which riders’ names were distinctively Black. Uber riders who use distinctively Black names experience up to 35\% longer wait times and more frequent cancellations than riders who use distinctively white names, especially for males in low population density areas.

These studies suggest that users of online platforms use visual and textual information to transfer their racial biases from the real world into the online world. Though difficult to confirm whether these stem from statistical or taste-based discrimination, they provide evidence that P2P market participants like Uber drivers, Prosper lenders, and iPod buyers use race information in their market decisions. 



\vspace{5mm}
\textbf{Research on Airbnb generally}

The appeal of easily accessible, affordable, and short-term peer-to-peer accommodation has led many travelers to opt for Airbnb over a hotel stay under the perception that a number of Airbnb attributes outperform those of traditional hotels \citep{guttentag}. This pattern has become increasingly prevalent in popular tourist cities, where Airbnb offers cheaper lodging closer to city centers than hotels \citep{gutierrez}. However, the influx of tourists into residential areas and lack of industry regulation on Airbnb are quickly becoming causes of economic and social concern. \cite{leong} investigate the legal ramifications of not subjecting Airbnb the same regulations as similar establishments. They note that the existing policies prohibiting public accommodations like hotels, restaurants, taxis, and retail businesses from discriminating against customers on the basis of characteristics such as race or religion have not yet evolved to apply to Airbnb, which exposes the platform to the risk of intentional bias and discrimination \citep{leong}. 

The rise of the Airbnb sharing economy has garnered interest in its microeconomic impacts and policy implications. \cite*{mao} investigated the effects of Airbnb on hotel prices in 221 different counties across the U.S. and found that the expansion of Airbnb into counties led to drops in hotel room demand, but also reduced unemployment in industries complementary to Airbnb’s core business, such as administration, support and waste management, and accommodation and food services. \cite*{barron} found evidence that a 10\% increase in Airbnb listings resulted in a 0.42\% increase in rents of residential properties, and a 0.76\% increase in house prices. These effects were found to be more pronounced in areas with smaller shares of occupiers who owned their place of residence, meaning that not only was the rent increase larger in these areas, but it also affected a larger share of those areas’ residents.



\vspace{5mm}
\textbf{Research on discrimination in Airbnb}

This study builds primarily on research done by \cite{edelman}, \cite{wang}, and \cite{kakar}. 

\cite{edelman} was the first to explore the effect of Airbnb host race on the price of a listing using a sample of 3,800 New York City hosts. They use Amazon Mechanical Turk workers to identify the race of the host in Airbnb profile pictures. Controlling for a listing's location, its rating on various dimensions, and several variables related to the size of the listing, the researchers find evidence that non-black hosts on Airbnb have prices roughly 12\% higher than black hosts.% 
	\footnote{I confirm \cite{edelman}’s results using my data in Table \ref{table:edelman_new}.} 
Their study was conducted on data from 2012, while Airbnb was relatively new, so they were able to include all of Airbnb's New York City listings in their sample. \cite{edelman} considered only Black hosts. They did not differentiate impact by gender, and were only able to control for a few property characteristics. 

Given these constraints, Wang et al. (2015) and Kakar et al. (2018) build on Edelman and Luca (2014)’s approach and apply it to study discrimination in the context of the Bay Area. 

Wang et al. (2015) analyzed the impact of Asian host race on Airbnb prices in Oakland and Berkeley by scraping information on 100 hosts from the Airbnb website in those areas. They found that when controlling for the number of bedrooms, bathrooms, and maximum occupancy, Asian hosts earned \$90 (20\%) less than white hosts with similar rentals. 

The fact that a price disparity is estimated for both Black and Asian hosts, albeit in different contexts, is potentially informative about the mechanism behind discrimination in this context. The Black-White wealth gap is well-documented, large, and pervasive [MELODY AND ABDALLAH cites]. When considering discrimination against Black hosts, it is unclear if guests discriminate because they expect listings of lower quality (statistical discrimination) or if they simply do not want to stay with Black hosts (taste-based discrimination). By contrast, Asians in the United States have the highest incomes out of any ethnicity in the US \citep{income}. It is therefore unlikely that discrimination against Asian hosts would be statistical. In fact, some studies have found evidence that discrimination against Blacks in rental markets is statistical in nature (that is, landlords use race as a proxy for income). For African-Americans who imply that they are of a higher social class when applying for an apartment, discrimination is virtually not present \citep{hanson}.




\cite{kakar} conduct a similar analysis for hosts in San Francisco as \cite{wang} in Oakland, and \cite{edelman} in New York City. They code the race, gender, self-identified sexual orientation, and whether the host is a couple from the host profile pictures. Since each picture required manual coding, they identify only 800 out of 6,000 active listings in San Francisco as of 2015. The researchers find that Asian hosts charge 8\% lower prices relative to White hosts for comparable listings, controlling for neighborhood property values from Trulia, area demographics from the Census, and occupancy rates purchased from AirDNA as proxies for desirability or attractiveness of the locations. The price disparity is 10\% for Hispanic hosts, but becomes insignificant when adding a control for occupancy rates. The price discrepancy for Asian hosts is persistent at 8\%. 

Several audit studies have also examined discrimination on the other side of the market - those who demand, rather than supply, listings on Airbnb. These studies follow the canonical model of the \cite{bertrand} study, which sent identical resumes to employers, varying only whether the names were canonically White or Black. 

\cite{cui} conducted an audit study to measure discrimination against Airbnb guests. They created fake guests accounts with identical profiles but with different names, either distinctively White or distinctively African American. In the first round of their experiment, they sent out requests for accommodation from a set of accounts with no reviews, as well as a set of accounts with one positive review. In the second round, the accounts were modified to have one negative review. They found that guests with White-sounding names were accepted on average 19 percentage points more often than those with African American-sounding names when the guest accounts had no reviews, but that the presence of a single review, whether positive or negative, rendered the acceptance rates statistically indistinguishable \citep{cui}.

In a more recent audit study, \cite{edelman2} measured discrimination against Airbnb guests. They created fake guest accounts that differed only by name and inquiring about the availability of listings across five cities. They found that requests for reservations by guests with distinctively African American names were 16\% less likely to be accepted by hosts than identical guests with distinctively white names. The estimated cost to the median host who rejects a guest on the basis of race was a loss of between \$65-\$100 of revenue \citep{edelman2}.

\vspace{5mm}
\textbf{Contribution to this literature}

I build on this literature in several ways. 

First, I bring more robust data to the question. My sample includes seven large urban centers that cover each geographic region in the US. This is important since discrimination in a large, cosmopolitan city with a highly diverse population such as New York might be different from a more racially homogeneous city, such as Nashville. 

Second, the comprehensiveness of the controls varies enormously between studies. \cite{wang} and \cite{edelman} are limited by the relatively sparse listing information available on the Airbnb website (Airbnb has continuously added more comprehensive listing information to their website). \cite{edelman}, for example, control only for a listing's location, its rating, and several variables related to the size of the listing. Since there could be a differences by race in the quality of the listing’s photos, or the descriptions on the listing page that could influence demand, including a more comprehensive set of covariates is important to more precisely measure any price disparity. To this end, I control for a comprehensive set of confounders that include all information available to a guest on the listing page. To account for unobservable difference in host quality, I use machine learning techniques to analyze the quality of the host-written descriptions.

Third, I contribute a reviewer-side analysis to the question. By coding demographic data for reviewers in Chicago, I am able to 

and conducting sentiment analysis on those reviews to give each review a quality score,

Since there is little variation in the numeric review score, the review score is uninformative for both potential guests and in terms of explaining listing quality. 

Fourth, in line with Becker’s theory of discrimination, I also test alternative hypotheses for these price disparities. Namely, FINISH HERE












\begin{comment}
Structurally, African-Americans were denied Federal Housing Adminstration mortgages at the low interest rates that were offered to white families, redlining districts , predatory , among many other structural moving north during the Great Migration were met with  Shut out from One reason is  
Many efforts have been made to curb discrimination in the housing sector against African-Americans. Landmark federal legislation such as the Fair Housing Act of 1968 prohibits housing discrimination based on race, the enforcement of anti-discrimination legislation is difficult on the local level. Residential preferences, differences in family structure and availability of affordable housing contribute to these disparities. Discrimination in housing has also been cited as one of the primary causes of these inequities. 

%Even though the accurate identification and measurement of discrimination by social scientists is vital to creating policies and statutes to combat it, measuring discrimination is difficult. Unobservable variables in the error term make it hard to isolate the effect of discrimination on the outcome variable of interest. Audit studies are one way that researchers can isolate the effect of race, sex, or other demographic on the outcome of interest. However, these types of experiments are not always possible due to the large organizational, manpower, or time costs associated with them. In the absence of an experimental set-up, regression models with a carefully chosen set of controls can aid in the accurate measurement of discrimination.  
Discrimination is difficult to measure. In the real world. Economists and other social scientists have long been concerned with the fact that minorities, especially African-Americans, have experienced pervasively lower living standards in the United States. One potential cause of this is discrimination in the housing market. 

Moreover, many small players have entered these markets who would have otherwise been unable to participate in traditional markets. Managing a room or home with Airbnb has much lower barriers to entry than being the landlord of a large apartment building. In just the 10 years since its founding, Airbnb has surpassed Marriott nearly three-fold in the number of rooms offered worldwide \citep{sharing}. 
\end{comment}

%As more people have entered these new markets, they have become increasingly dependent on the supplementary income they provide. Hosting with Airbnb, a platform that allows people to rent out their apartment, house, or single room to short-term lodgers, is one such opportunity. A 2017 report released by Airbnb states that in rural areas, hosts get as much as 5 - 20\% of their income from their listing \citep{rural}. Airbnb's fastest growing demographic of hosts, women over 60 years of age, earn \$6,000 a year on average from hosting, often relying on that income to supplement retirement savings \citep{elderly}\citep{nyt2}. 

%Some residents of areas of New York City have started relying on hosting with Airbnb to pay for rent or fund retirement \citep{nyt1}. 

%The extent to which hosts have grown to rely on Airbnb as a source of income makes discrimination on the platform a relevant topic of research. The economic consequences of discrimination are substantial - hosts who are discriminated against would face lower demand, have higher vacancies, and earn less revenue from their listing. While one previous paper found evidence of discrimination against New York City hosts using data from 2013, no other more recent or comprehensive research has been done on this type of discrimination on Airbnb. 

%In this paper, I empirically investigate the existence and extent of anti-host discrimination in Airbnb. I start by measuring the effect of host race and sex on the price of the listing and on a constructed measure of host revenue. I use data from a webscrape of around 70,000 Airbnb listings across 7 U.S. cities.\footnote{The scrape includes all of the property, host, and review information on a listing profile. To see what information would be available, see Figures 1-5 for screenshots of a sample listing. All of the information seen on the sample listing is included as variables in the data set.} For each of the 70,000 listings, the race, sex, and age of the host from their profile picture was coded. 

%Next, I construct a measure of host revenue by multiplying the price a host charges by the total number of reviews for that listing (a proxy for the quantity demanded). Using this measure of revenue, I estimate that White female hosts, Black male hosts, Black female hosts, and Asian female hosts lose about \$100-\$300 in revenue over the course of a year as compared to White male hosts who own similar listings. The exact revenue loss depends on the coefficients on price and number of reviews of a particular host.\footnote{See Table 5 and Section 3.2 for the exact effects on revenue.} These effects are statistically significant at the p $<$ .05 level or higher, and significant at the p $<$ .001 level for White females and Black females. There are also negative effects on revenue for Hispanic hosts and Asian males, but they are not significant. In Section 4, I also conduct several robustness checks and show that these results hold across various cities, price ranges, time on the market, and property types.\footnote{See Tables 6, 7 and discussion in Section 4.}

% Understanding discrimination in this new housing market is important because it ties into racial discrepancies in housing widely observed by economists and other social scientists. 

% However, it is difficult to separate the effect of current racial discrimination from the confounding effect of these other economic realities. It is therefore unclear to what extent current discrimination, especially in the housing market, contributes to these long-standing economic disparities. 

%Even though the accurate identification and measurement of discrimination by social scientists is vital to creating policies and statutes to combat it, measuring discrimination is difficult. Unobservable variables in the error term make it hard to isolate the effect of discrimination on the outcome variable of interest. Audit studies are one way that researchers can isolate the effect of race, sex, or other demographic on the outcome of interest. However, these types of experiments are not always possible due to the large organizational, manpower, or time costs associated with them. In the absence of an experimental set-up, regression models with a carefully chosen set of controls can aid in the accurate measurement of discrimination.  

% Economists and other social scientists have long documented the poor housing outcomes for minorities, particularly African-Americans, in the housing market.