howdy
African-Americans have experienced pervasively worse outcomes in the housing market, and historic and current racial discrimination is one major cause \citep{krysan}. Even after the gains of the Civil Rights Era, such as the landmark Fair Housing Act of 1968, discrimination in the housing market is widely documented by social scientists. \cite{yinger1} measured discrimination using fair housing audits and found that black renters are told that there are 30\% fewer available housing units than white renters. African-American families face higher barriers when raising capital to purchase a home, since black applicants get fewer loans for the same application as white applicants \citep{pope}. Emails sent to landlords from home-seekers with African-American-typical names receive lower response rates than emails sent by those with names commonly associated with whites \citep{hanson}.%
	\footnote{Some studies have found evidence that discrimination in rental markets is statistical in nature (that is, landlords use race as a proxy for income). For African-Americans who imply that they are of a higher social class when applying for an apartment, discrimination is virtually not present \citep{hanson}.}

%\citep{yinger1}
%Landlords renting out apartments discriminate both because of their own prejudice and in response to the prejudice of their white renters \citep{ondrich}.

Economists have primarily studied discrimination against African-American tenants. There is little research on the other side of the market - when African-Americans are supplying, rather than demanding, housing.%
	\footnote{One would also expect black landlords to fare worse than white landlords in this area as well. Properties owned by African-Americans tend to be less expensive than those owned by white Americans. The average black household still has less mean wealth than a white household \citep{oliver}. Even middle-class black and Hispanic households still live in neighborhoods with median incomes similar to those of very poor white neighborhoods \citep{reardon}.} 
Because property ownership cannot be randomized, it is difficult to disentangle true discrimination from systematic differences in the housing owned by African-Americans and white landlords. Some studies have found anecdotal evidence of discrimination against African-American landlords. For instance, the spike in subprime lending and the ensuing foreclosure crisis was found to be causally linked to residential segregation, and evidence suggests that specifically black residential dissimilarity and spatial isolation were important predictors of foreclosures across U.S. metropolitan areas \citep{foreclosure}.

This paper leverages these recent rise of sharing economies and the data and standardization their platforms provide to credibly measure discrimination against minority landlords. Airbnb is a sharing economy platform that allows people to rent out their apartment, house, or single room to short-term lodgers. Since Airbnb simply provides an online platform for a market that already exists, it is reasonable to assume that agents will discriminate on Airbnb in a similar way that they discriminate in the real world. This means that studying discrimination in sharing economies could be an important way to learn about discrimination in traditional housing markets as well.
%Sharing economies are convenient for research in several distinct ways. First, all agents on the platform obey a uniform set of rules and observe the same set of information about one another. Second, sharing economies allow for cleaner data collection, as well as both price and a proxy for quantity demand for each listing. 

Measurements of discrimination on Airbnb are still potentially confounded by many other factors that affect a listing's price. Edelman and Luca (2014) estimated the effect of host race on the price of their listing in New York City with a small set of controls. Their results suggest that non-black hosts on Airbnb have prices roughly 12\% higher than black hosts. However, they control for only a few property characteristics, the quality of the host's reviews, and a measure of the reliability of the host. They leave out many other unobservables that could be confounding the effect such as the type of listing (which is important if black hosts own single rooms and white hosts own entire houses) and proxies for the quality of the host themselves (which could also vary by race). 

In this paper, I use a new, previously unexploited dataset from a webscrape of Airbnb to empirically measure discrimination. Using price and quantity information for 70,000 Airbnb hosts throughout the country, I address the limitations of Edelman and Luca's research by controlling for location, additional property characteristics, comprehensive measures of listing size, and text analyses of host-written descriptions of the listing. 

I find that non-white hosts, both male and female, have lower prices than white hosts. The biggest effect is for Asian female hosts, whose prices are roughly \$9 less per day than white male hosts who own the same type of listing. The second biggest effect is for black males, whose price is lower by \$7, followed by black women and Asian men with price disparities of \$6 per day, and Hispanic females with a coefficient of \$5.%
	\footnote{This effect is statistically significant at the p $<$ .001 level for black hosts and Asian women, the p $<$ .01 level for Asian male hosts, and p $<$ .05 level for Hispanic women.} 
For Hispanic men the effect is small, around \$2, and is not statistically significant. 

I have data not just on the price, but the number of reviews and the vacancy rate of a listing, two measures of the quantity demanded. This is important because knowing both quantity demaned and price allows me to distinguish whether the price disparity between minority and white hosts on Airbnb is due to a demand shift (consistent with discrimination) or a supply shift (consistent with a difference in marginal cost for the hosts, or other hypothesis). 

Both measures of quantity demanded - the number of reviews and the vacancy rate - confirm that despite lower prices, minority hosts have a lower quantity demanded. 


Quantity demanded measures are also lower for minority hosts - black females receive half as many reviews as white males, and the coefficients for black males and Asian hosts are less statistically significant, but are also negative, indicating fewer reviews for their properties as well.




My first measure of quantity demanded is the number of reviews. In Table 7, I regress the number of reviews on host race, controlling for the same set of models as Table 5. I find that minority hosts have either the same or lower review numbers than white hosts for a listing that spent the same amount of time on the market. Most coefficients are either roughly zero, or in the range of 1-2 reviews less than white hosts. In general, my results suggest that minority hosts see a lower quantity demanded than white male hosts.%
\footnote{It is important to keep in mind that this conclusion is only salient if the total number of reviews is a reasonable proxy for the demand of a listing. Yet, one can imagine that if reviewers systematically under-review minority hosts relative to white hosts, these groups would have lower numbers of reviews that do not necessarily represent a lower quantity demanded. There is no way to tell apart these mechanisms in my data, and no previous research has been done on race-based differences in review rates. My working assumption is that even if not every guest leaves a review, the review proportion is similar across host race, and a lower number of reviews therefore indicates a real difference between quantity demanded of minority hosts and white hosts. }





 To test this hypothesis, I consider the number of reviews that a host has as a proxy for the quantity demanded of that listing. I regress the number of reviews on host race, controlling for the preferred specification. I find that minority hosts have a lower, not higher, number of reviews than white hosts, suggesting that this hypothesis fails to explain the price disparity.%
	\footnote{Throughout this analysis, I assume that guests review hosts of different race at equal rates. A recent study found that reviews left by hosts on guests’ pages can significantly reduce discrimination and render acceptance rates of guests with White-sounding names and African American-sounding names statistically indistinguishable \citep{cui}; however, no studies have been conducted as of yet to investigate the possibility of racial bias on guests’ part towards hosts through the mechanism of the online reputation system, and so it remains unknown whether or not reviewers discriminate against minorities in leaving reviews \citep{ye}. However, if reviewers systematically under-review minority hosts, this itself could be evidence of discrimination.} 

One potential explanation for their lower number of reviews is that minority hosts make their listing available to guests less frequently than white hosts. A host controls how many days of the month they offer their listing for rent via an availability calendar on the listing page. When a guest books their listing, the booked days disappear from the availability calendar. Therefore, the measure of availability is actually a measure of true vacancy for the listing. If minority hosts have lower numbers of reviews, perhaps this is because they offer their listing for fewer days of the month than white hosts. To test this, I regress a listing's availability out of 30 days on host race, controlling for the preferred specification. Contrary to this hypothesis, regression results show that the listings of black hosts stay vacant 1 - 2 days per month longer than the listings of white hosts. However, there is evidence that Asian hosts do choose to make their listings available less frequently, which would contribute to their lower number of reviews. 






I also conduct several robustness checks of this result. I find that prices are lower for black hosts across all cities in my sample, while prices for Hispanic and Asian hosts are generally lower as well (with a couple exceptions in cities with a small sample size of Hispanic and Asian hosts, such as Nashville and DC), indicating that the price disparity is not driven by a single city. Secondly, I break up my sample by various listing characteristics (price, type of property, and location), and find that the price disparity holds across all types of listings. 

Thirdly, I investigate whether or not the price disparity is driven by minority hosts owning listings of worse quality, or simply being worse hosts, than whites. I consider the quality of a host's reviews as a proxy for the quality of the listing and host. I use the race and gender of the reviewer and the host to compare the sentiment (how favorable or unfavorable the review is) of the reviews that guests leave for white and for minority hosts.%
	\footnote{Since it required hand-coding, demographic information of the reviewers is only available for a randomly-chosen subset of hosts in Chicago.} 
Rather than observing that minority hosts uniformly had lower quality reviews, which the hypothesis would predict, the significance of the result was either negligible, or depended on the demographics of the reviewer and host. While there is some evidence that male reviewers tend to rate male hosts higher, there is little within-race preference between reviewers and hosts. Taken as a whole, sentiment analysis suggests that minority hosts do not have lower quality reviews. 

\subsection{About Airbnb} 
Airbnb is an online marketplace founded in 2008 that allows hosts to rent their private dwellings to guests as temporary accommodation. As of 2017, it has more than 3 million listings, more than Marriott's 1.2 million rooms worldwide \citep{aboutus}. Just like traditional hotel chains, guests on Airbnb can browse listings by city and property type, and book a stay based on prices, location, past reviews, pictures of the listing, size, and amenities. Unlike traditional hotel chains, however, hosts create a profile for themselves and a page for each listing they are renting. Each listing page includes the name and picture of the host, the reviews left by previous guests, and those guests' profile pictures. Guests can therefore infer demographic information about the host through a host's picture and name, creating the opportunity for discrimination. Figures 1-2 present screenshots of a listing in a Chicago neighborhood, illustrating some of the information that would be available to a potential guest. Figures 3-6 display an example of reviews, a sample host profile, and sample location maps, and can be found in the Appendix.


\subsection{Previous Literature} 
Most relevant to this paper is \cite{edelman}, the first paper to identify and measure the extent of anti-host discrimination on Airbnb. They explore the effect of host race on the price of their listing using a snapshot of roughly 3,800 New York City hosts in 2012. Controlling for several confounders that influence price, their findings indicate that non-black hosts on Airbnb have prices roughly 12\% higher than black hosts. I build on Edelman and Luca's research in several important ways. First, their sample was relatively narrow and confined to a single city. My sample includes seven large urban centers that cover each geographic region in the US. INSERT MORE DETAILS ABOUT REPLICABILITY OF SAMPLE. Discrimination in a large, cosmopolitan city with a highly diverse population such as New York might look different from discrimination in Nashville, which is more racially homogenous.%
	\footnote{MOVE DEMOGRAPHIC DETAILS TO DATA APPENDIX. According to the U.S. Census Bureau, Nashvile is 60.4\% white, 28.4\% black, 10.0\% Hispanic, and 2.5\% Asian. New York City is 44\% white, 25.5\% black, 12.7\% Asian, and 28.6\% Hispanic.} 
Second, their set of controls was limited by the relatively sparse listing information available on the Airbnb website (Airbnb has added more comprehensive data on each listing page since 2013). Their covariates only included a listing's location, the number of people the listing accommodates, the rating, the number of bedrooms, and whether or not the whole apartment is available to the guest. After confirming their results with their model, I then control for a more complete set of covariates (see Section 2.2 and Section 3.1).%
	\footnote{See Table 6 for the results of my regression using their covariates. See Section 3.1 for a discussion of my controls.} 
Most importantly, I also test alternative hypotheses for these price disparities, which Edelman and Luca are unable to address. 

\cite{becker} proposed the idea that discrimination against a group can be reflected in that group's market prices. In the Airbnb market, Becker's market discrimination would be reflected in the price that the guest (buyer) pays to the host (seller) to stay with them. If the guest (buyer) is discriminating, then given two comparable listings, they would choose not to stay in the one owned by a minority host (seller). Responding to the lowered demand curve, hosts in minority groups rationally post a lower price and, despite this, face a lower quantity demanded. 

Becker (1957) was concerned with discrimination arising from face-to-face interactions between minority and majority groups. Since then, there has been a large amount of research indicating that Becker's theory holds for people participating in online markets for labor, lending, rental, and products. In these cases, participants simply bring their prejudices online and use names and photos to discriminate. A canonical example is the \cite{bertrand} study, which found that resumes with white sounding names received 50\% more callbacks from potential employers than identical resumes with African-American sounding names. \cite{doleac} examined market outcomes when selling an iPod on various online marketplaces. In some pictures, a dark-skinned hand was holding the iPod, signaling a black seller, while in others, a light-skinned hand was holding the iPod, signaling a white seller. Hands which indicates a black seller received fewer and lower offers than white sellers. In sharing economies, a similar pattern occurs. Uber riders with distinctively African-American names experience longer wait times and more frequent cancellations than riders who use distinctively white names \citep{knittel}. A later study by \cite{edelman2} found a similar result: guests with distinctively African-American names receive 16\% fewer responses from Airbnb hosts than those with white names. These examples suggest that users of online platforms transfer their biases from the real world into the online world.  

\begin{comment}
Structurally, African-Americans were denied Federal Housing Adminstration mortgages at the low interest rates that were offered to white families, redlining districts , predatory , among many other structural moving north during the Great Migration were met with  Shut out from One reason is  
Many efforts have been made to curb discrimination in the housing sector against African-Americans. Landmark federal legislation such as the Fair Housing Act of 1968 prohibits housing discrimination based on race, the enforcement of anti-discrimination legislation is difficult on the local level. Residential preferences, differences in family structure and availability of affordable housing contribute to these disparities. Discrimination in housing has also been cited as one of the primary causes of these inequities. 

%Even though the accurate identification and measurement of discrimination by social scientists is vital to creating policies and statutes to combat it, measuring discrimination is difficult. Unobservable variables in the error term make it hard to isolate the effect of discrimination on the outcome variable of interest. Audit studies are one way that researchers can isolate the effect of race, sex, or other demographic on the outcome of interest. However, these types of experiments are not always possible due to the large organizational, manpower, or time costs associated with them. In the absence of an experimental set-up, regression models with a carefully chosen set of controls can aid in the accurate measurement of discrimination.  
Discrimination is difficult to measure. In the real world. Economists and other social scientists have long been concerned with the fact that minorities, especially African-Americans, have experienced pervasively lower living standards in the United States. One potential cause of this is discrimination in the housing market. 

Moreover, many small players have entered these markets who would have otherwise been unable to participate in traditional markets. Managing a room or home with Airbnb has much lower barriers to entry than being the landlord of a large apartment building. In just the 10 years since its founding, Airbnb has surpassed Marriott nearly three-fold in the number of rooms offered worldwide \citep{sharing}. 
\end{comment}

%As more people have entered these new markets, they have become increasingly dependent on the supplementary income they provide. Hosting with Airbnb, a platform that allows people to rent out their apartment, house, or single room to short-term lodgers, is one such opportunity. A 2017 report released by Airbnb states that in rural areas, hosts get as much as 5 - 20\% of their income from their listing \citep{rural}. Airbnb's fastest growing demographic of hosts, women over 60 years of age, earn \$6,000 a year on average from hosting, often relying on that income to supplement retirement savings \citep{elderly}\citep{nyt2}. 

%Some residents of areas of New York City have started relying on hosting with Airbnb to pay for rent or fund retirement \citep{nyt1}. 

%The extent to which hosts have grown to rely on Airbnb as a source of income makes discrimination on the platform a relevant topic of research. The economic consequences of discrimination are substantial - hosts who are discriminated against would face lower demand, have higher vacancies, and earn less revenue from their listing. While one previous paper found evidence of discrimination against New York City hosts using data from 2013, no other more recent or comprehensive research has been done on this type of discrimination on Airbnb. 

%In this paper, I empirically investigate the existence and extent of anti-host discrimination in Airbnb. I start by measuring the effect of host race and sex on the price of the listing and on a constructed measure of host revenue. I use data from a webscrape of around 70,000 Airbnb listings across 7 U.S. cities.\footnote{The scrape includes all of the property, host, and review information on a listing profile. To see what information would be available, see Figures 1-5 for screenshots of a sample listing. All of the information seen on the sample listing is included as variables in the data set.} For each of the 70,000 listings, the race, sex, and age of the host from their profile picture was coded. 

%Next, I construct a measure of host revenue by multiplying the price a host charges by the total number of reviews for that listing (a proxy for the quantity demanded). Using this measure of revenue, I estimate that White female hosts, Black male hosts, Black female hosts, and Asian female hosts lose about \$100-\$300 in revenue over the course of a year as compared to White male hosts who own similar listings. The exact revenue loss depends on the coefficients on price and number of reviews of a particular host.\footnote{See Table 5 and Section 3.2 for the exact effects on revenue.} These effects are statistically significant at the p $<$ .05 level or higher, and significant at the p $<$ .001 level for White females and Black females. There are also negative effects on revenue for Hispanic hosts and Asian males, but they are not significant. In Section 4, I also conduct several robustness checks and show that these results hold across various cities, price ranges, time on the market, and property types.\footnote{See Tables 6, 7 and discussion in Section 4.}

% discrimination is hard to measure, 
% Understanding discrimination in this new housing market is important because it ties into racial discrepancies in housing widely observed by economists and other social scientists. 

% However, it is difficult to separate the effect of current racial discrimination from the confounding effect of these other economic realities. It is therefore unclear to what extent current discrimination, especially in the housing market, contributes to these long-standing economic disparities. 

%Even though the accurate identification and measurement of discrimination by social scientists is vital to creating policies and statutes to combat it, measuring discrimination is difficult. Unobservable variables in the error term make it hard to isolate the effect of discrimination on the outcome variable of interest. Audit studies are one way that researchers can isolate the effect of race, sex, or other demographic on the outcome of interest. However, these types of experiments are not always possible due to the large organizational, manpower, or time costs associated with them. In the absence of an experimental set-up, regression models with a carefully chosen set of controls can aid in the accurate measurement of discrimination.  

% Economists and other social scientists have long documented the poor housing outcomes for minorities, particularly African-Americans, in the housing market.