\subsection{Main result: Do minority hosts have lower prices than white hosts?} 

Before analysis, the data set used was restricted to hosts who have profile pictures and manage less than 20 listings, and listings priced at less than \$800 per night. 64,611 listings were left after restricting the data set. There were only 20 hosts who did not have profile pictures.

Table 3 presents OLS estimates of the effect of host race and gender on the listing price. The specification is of the form: 

\[ Price_{i,j} = \beta_1 Race_{i}\,X \,Sex_i + \beta_2 Age_i + \beta_3 x_{i,j}\]

The $Price_{i,j}$ is host $i$'s price from their Airbnb listing $j$. For hosts with multiple listings, each listing is treated separately. The $Race_{j}\,X \,Sex_j$ is the interaction of the race and sex of the host. White males are the omitted category. $Age_i$ is the age of the host (young, middle-aged, or senior). Young hosts are the omitted category for age. $x_{i,j}$ is vector of other covariates that grows from left to right in the columns of Table 3. The columns are additive in their covariates, so each column controls for everything in the previous columns, plus a new set of covariates. Standard errors are clustered by neighborhood throughout.

The first column, Model 1, in Table 3 presents the raw effect of host race and sex on the price of a listing. These are consistent with the mean listing prices by race presented in Table 2, except now also broken down by male and female hosts within each racial category.

Model 2 adds city and neighborhood fixed effects.%
	\footnote{Neighborhoods are assigned in accordance with each city's designations. In Chicago, for example, the fixed effect granularity is at the level of locating a listing within Hyde Park vs. Woodlawn (The University of Chicago is located in Hyde Park, and Woodlawn is directly adject to it).}
Location is an important proxy for income levels, crime rates, and distance from downtown, which are all predictors of property prices and therefore listing prices on Airbnb. As expected, controlling for location substantially decreases the estimated racial gaps in prices. The coefficients for minority hosts decrease from a range of \$20-40 to a range of \$10-20 (these are all negative, and relative to white male hosts). I observe the largest decrease in the coefficients on black hosts, which go down from \$40 to roughly \$15. Coefficients of Hispanic hosts decrease by around \$10; Asian hosts by about \$20. 

It is well-documented that blacks in urban populations are nearly four times more likely than whites to live in neighborhoods where the poverty rate is 40\% or higher.\cite{firebaugh} In fact, minorities at every income level live in poorer neighborhoods than do whites with comparable incomes. For example, a black household earning \$75,000 a year resides in a higher-poverty neighborhood than a white household with earnings of less than \$40,000 a year.\cite{logan} It is therefore expected that a large part of the variation in Airbnb prices between those groups can be explained by their listing's location. The coefficients of white females, on the other hand, persist at around \$4 with the addition of location controls. This is most likely because white females tend to live in the same areas as white males and therefore have little to no variation in price that can be explained by differences in neighborhood.  

Model 3 adds controls for listing-specific characteristics. Listing characteristics include fixed effects for the property type and room type, the listing's duration on the market, the number of guests the listing accommodates, the number of bathrooms, bedrooms, and beds, the bed type, the number of amenities, the number of minimum nights, any extra fees, whether the listing is instantly bookable, and the cancellation policy.% 
	\footnote{The listing's duration on the market is proxied by fixed effects for the month and year of the listing's first review.}
Controlling for these listing characteristics decreases all effects to \$5-10, depending on the race of the host. Asian female hosts have the largest decrease in coefficient after controlling for listing characteristics, which indicates that a substantial part of their effect is driven by owning properties with worse observable characteristics. The effects on middle-aged and senior hosts are almost eliminated by controlling for property characteristics, indicating that their higher listing prices are primarily driven by better observable characteristics. The effects for Hispanic males and white women largely disappear with the addition of property controls. 

In general, from Model 1 to Model 3, coefficients steadily decrease in magnitude and the $R^2$ increases from .166 with neighborhood controls to .621 with listing controls. Most of the variation in price between minority hosts and white male hosts can be explained by either the property's location or observable property characteristics. The $R^2$ jumps substantially to .621 in Model 3, so adding property characteristics explains much more of the total price variation than the location. This might be because Airbnb listings tend to be more concentrated in certain areas of each city (North Side in Chicago, lower and middle Manhattan in New York City, etc). If listings in a city cluster together instead of being uniformly dispersed, then controlling for location won't explain as much of the variation as controlling for property characteristics. LISTING OR PROPERTY PICK ONE - FIX

Model 4 in the last column presents my full, preferred specification. It adds host-specific characteristics to Model 3, including the host response time and the host response rate, whether the host is a Superhost, whether the host identity was verified by Airbnb, and if the host requires a guest's profile picture or phone to book. 

Importantly, Model 4 also controls for variation in host effort. I attempt to account for the idea that some hosts may have higher prices not because of better observable characteristics, but just because they are better hosts. There are several host-written fields on each listing page, the ``Summary", ``Description", ``Space", ``Neighborhood Overview", ``Transit", and ``Notes". By filling out these fields, hosts not only describe their listing, but have the opportunity to provide guests with helpful tips and information about the surrounding area. How well a host writes these descriptions is an indication of how much effort they are willing to put into hosting. To this end, I construct three variables to measure host effort. My first variable simply measures the length of each of these fields. Presumably, the longer the description, the more effort the host put into writing it. My second variable measures whether these fields had mostly long words or short words, so that a description that uses shorter words, such as ``My house is nice", would be counted as lower quality than ``My house is gorgeous". 

My third measure of host effort is a rudimentary sentiment analysis of the ``Description" field. Hu and Liu (2004) create a list of 2,006 positive words that commonly appear in customer reviews to aid in sentiment categorization.\cite{hu} I only include words that have substantial variation in the description, meaning that more than 5\% of descriptions had these words. This narrowed the list of viable words significantly. I take 7 positive words from that list that would be most relevant for Airbnb listings: ``spacious", ``beautiful", ``clean", ``comfort", ``great", ``love", and ``quiet". I then added a covariate for the number of these ``good words" in the host's ``Description" field. Together, these three ``host effort" variables control for hosts who write longer descriptions, use longer words in those descriptions, and put more words that are associated with positive reviews in their descriptions. 

After controlling for my final specification, I estimate that, across the board, minority hosts earn lower prices from their Airbnb listing than white hosts. The biggest effect is for Asian female hosts, whose prices are roughly \$9 less per day than white male hosts who own the same type of listing. The second biggest effect is for black males, with a coefficient of \$7. The coefficients on black women and Asian men are \$6 per day each, Hispanic females is \$5. This effect is statistically significant at the p $<$ .001 level for black hosts and Asian women, the p $<$ .01 level for Asian male hosts, and p $<$ .05 level for Hispanic women. For Hispanic men the effect is around \$2 and is not statistically significant. There is little effect for white females, and a small effect that's not statistically significant for middle-aged and senior hosts. An F-test shows that host race is jointly significant for price at the p $<$ .001 level after controlling for both property and host characteristics. CHECK THIS FTEST My results are stable to the addition of host characteristic controls while still clustering standard errors at the neighborhood level. The inclusion of these host characteristics does not improve the fit of the model substantially. Property characteristics and location still explain more of the variation in price than host characteristics. 

My results are consistent with Edelman and Luca's findings, but I find smaller effects (they found a 12\% price disparity, I found about a 7\% price disparity). This is most likely because I control for a larger set of covariates. To confirm this, I run a regression on listings in New York City, controlling for the same covariates that Edelman and Luca used in their main result. The results, presented in Table 4, show that I get the same coefficient as the one they found - an \$18 (X\%) price difference between black hosts and white hosts. This indicates that my main results in Table 3 were smaller because I controlled for more variation, not because of a structural change in the extent of discrimination in Airbnb.%
	\footnote{Airbnb has changed their user interface in the past four years, so I approximated several of their regressors with the closest variable available in my data. For example, instead of whether the host had social media accounts, I controlled for whether the host's contact information was verified by Airbnb.}

If one believed the price difference was driven by unobserved characteristics, one might have expected that the price gap between white and minority hosts would disappear with the addition of more controls. This is true up to a point, since when I add more covariates my coefficients shrink relative to Edelman and Luca's. However, after that, my coefficient of interest is stable to the addition of controls - adding host-specific controls does not substantially change any of the effects. As one might expect, the $R^2$ goes up to .621 with the addition of location and property controls, but adding my host controls increases the $R^2$ by only .006. 

There are a few possible sources of unexplained variation in the price of the listing - variation in the real, physical qualities of the listing that wasn't captured by the property controls, and variation in the quality of the listing's profile that was not captured by the host controls. Since I was able to control for all of the property-quality variables that Airbnb offers on a listing page, it is unlikely that there are unobserved property characteristics driving the price differences. Since adding host controls explained very little variation in the price, increasing the $R^2$ by only .006, it is unlikely that adding more sophisticated measures of host quality or effort would significantly help explain price disparities. While this does not eliminate the possibility that there is a set of controls not related to property type or host type that would have increased the $R^2$ drastically, this is still good evidence to believe that the price difference I estimate is a real difference, rather than purely caused by endogeneity. 


\subsection{Secondary result: Estimate of yearly revenue loss for minority hosts}
MOVE THIS TO CONLUSION!

To estimate the revenue loss that would result from the price differences found in the previous section, I construct a measure of revenue equal to: 

\[Total \: Revenue \ = \ Price \: Per \: Day \ * \ Reviews \: Per \: Month \ * \ 12\] 
%% \ and \: and \, and \quad and \qquad symbols make spaces in math mode

The estimates of the effect of host race on host revenue are in Table 5.\footnote{I acknowledge that there are potential problems associated with using number of reviews as a proxy for demand, which are briefly mentioned in the introduction. I fully discuss them in Section 5.} Consistent with my prediction, all of the estimates are negative and in the range of \$100-300 dollars. The biggest yearly revenue loss in the entire sample is for black females at \$300, or about a 12\% loss. Black males and Asian women lose about \$160-180 throughout the year. Notably, white females, who had no statistically significant effects on their price in Table 3, have a significant revenue coefficient of \$144. This is because of reasons inherent to the definition of revenue. Even if white women didn't have a statistically significant difference in price from their male counterparts, they do have a lower number of reviews. In the next two sections, I present and discuss evidence that \textit{both} the price and the number of reviews for minority hosts are lower than for white male hosts. That means that when I multiply these two values together, two complementary effects both lower the total revenue of hosts. Overall, however, the same groups which had significantly lower prices also have lower revenues - black males, black females, Asian females, and white females all have significant effects in the range of several hundreds of dollars. 
 


\section{Robustness checks of main result} %%%%%%%%%%%

In this section, I see if the effects on price persist when I break up the data by city and property type. 

\textbf{Effects persist between large cities}

Edelman and Luca found much larger effects of host race just in New York City than I did in data that included all seven cities. A reasonable hypothesis for this is that New York City is driving all of the variation in price, and when other cities are included, where discrimination might not exist, the effects get smaller. To test this, I broke up the effects of host race on listing price by city and controlled for my preferred specification. The results are in Table 6. In general, no single large city in my data set is driving all of the variation in my data. The effects on price are mostly negative for minority hosts, with a few positive coefficients in cities with fewer observations. As expected, New York City and Los Angeles, the cities with the most host diversity and largest sample size, most closely resemble the coefficients from my main result in Table 3. In smaller cities, more than half of the negative effects are significant to various levels, and none of the positive effects are significant. 

However, there are a few outlier coefficients that are most likely driven by low sample size in smaller cities. The coefficients for black hosts are fairly consistent with the combined data in all cities but New Orleans. In New Orleans, a black host is estimated to earn \$18 less for the same kind of listing as a white host, an effect that is statistically significant at the p $<$ .05 level. The coefficients on Hispanic hosts are mixed - in LA, NYC, and Chicago, the coefficients on Hispanic hosts are approximately the same as the combined analysis, while in Austin, New Orleans, and DC, the coefficients are slightly positive. The outlier coefficient is in Nashville, where Hispanic hosts are estimated to earn \$39 less per day than a comparable white host. However, there are only 21 Hispanic hosts in Nashville, so this result is not very generalizable. In LA and NYC, the coefficients of Asian hosts are consistent with the combined data in sign and magnitude. However, they have large coefficients of -\$18 to \$28 in Chicago and Austin, respectively, both of which are significant. The reasoning is similar to Hispanic hosts. 

Generally speaking, the price difference in New York City and Los Angeles is relatively the same. While there are outlier coefficients in smaller cities, it is unlikely that discrimination against Asian and Hispanic hosts in those cities is actually 8 times higher than New York City or Los Angeles. Rather, those cities often have less than 50 hosts in a particular racial category, so any outliers have the potential to skew the coefficients to a much larger degree.  

\textbf{Effects persist between listing types}

Table 7 presents the effects of host race on price, broken down by various listing characteristics such as price, time on market, and property type. I break up the listings by price in the following way: separately for Los Angeles and New York City, I predict the price of a listing based on its property and host characteristics, without host race. I then use this predicted price to break up listings into higher than, and lower than, the mean predicted price in each city. I control for my preferred specification.  

I find that there is much greater price disparity between white and minority hosts among high-priced listings rather than low-priced listings. Column 1 of Table 7 considers price disparities only in listings priced below-average in Los Angeles, and Column 2 considers only above-average priced listings in Los Angeles. The price disparities are much larger for the expensive listings - \$10 for black hosts, \$15 for Hispanic hosts, and \$18 for Asian hosts. By contrast, the coefficients are much smaller, only \$2-3, for the cheaper listings. This pattern in price disparities indicates that discrimination is more pronounced against minority hosts who own more expensive properties than against minority hosts who own cheaper properties. In New York City (Columns 3 and 4), the coefficients for expensive listings are larger than coefficients for cheap listings by about \$6, so this effect is not limited to one city or driven exclusively by city-specific characteristics. One hypothesis for this effect is that all of the discrimination is statistical, in which case the host race isn't as much as a proxy for property value for guests for cheap listings  as it is for expensive listings, which are owned primarily by white hosts. Another explanation is that if a guest is expecting to put up a larger financial investment, they are more selective about which listings they stay at, so any existing discrimination is exacerbated. 

Columns 5 and 6 show that the price disparities are similar for both old and new listings (old listings are defined as those which have been on the market for more than two years). This suggests that the effects of discrimination are \textit{not} erased out by spending a longer time on the market; in other words, minority hosts do not simply ``catch up" to white hosts after a few years. Columns 7-9 break price disparities up by property type. The effects for black hosts across property types get more pronounced the more expensive the property type. While black hosts who own an apartment expect to earn \$5 less per day than white hosts, that number increases to an \$11 loss for black hosts who own houses. This is consistent with the results in Columns 1-4 and with the statistical discrimination hypothesis - if hosts are using race as a proxy for property value, then we should expect guests to discriminate more against minority hosts who own more expensive properties, such as houses.   