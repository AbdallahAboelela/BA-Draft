
Long after the passage of anti-discrimination laws in the housing sector, pervasive disparities remain between minority and white landlords. In the absence of an experiment that randomizes property ownership, it is difficult for economists to measure the extent of discrimination empirically. In this paper, I calculate the cost of discrimination to minority hosts by estimating the minority-white price disparity in the short-term housing market of Airbnb. I find that black and Asian hosts earn \$5 - \$9 less per day for the same type of listing as white hosts. I overcome the usual hurdles in measuring discrimination by using both price and quantity demanded information of hosts' listings, concluding that discrimination is the most likely cause of the price disparity for black hosts. 

Since prices only matter to hosts to the extent that they affect revenue, in this section I also estimate the impact of the price disparity on hosts' annual revenue. To estimate the revenue loss that would result from the price differences found in the previous section, I construct a measure of revenue equal to: 

\[\text{Total revenue} \ = \ \text{Price per day} * \text{Reviews per month}  * 12\] 
%% \ and \: and \, and \quad and \qquad symbols make spaces in math mode

The estimates of the effect of host race on host revenue are in Table \ref{table:revenue_new}. Consistent with my prediction, all of the estimates are negative and in the range of \$100 - \$300 dollars. The biggest yearly revenue loss in the entire sample is for black females, who could expect to earn \$300 per year less, or about 12\%, than a white male operating the same listing. Black males and Asian women would lose about \$160 - \$180 throughout the year. White females, who have no statistically significant effects on their price, have a significant \$144 loss in revenue. This is best explained by their lower number of reviews. 

Airbnb itself can do much to address issues of discrimination on the platform. In response to media outcry about allegations of discrimination, Airbnb updated its Discrimination Policy in September 2016, increasing instant bookings (the opportunity for guests to book without waiting for host approval) and making host profile pictures smaller. Evaluating Airbnb's efforts to address discrimination is therefore a relevant extension of this research. Since InsideAirbnb.com is continually being updated, there is now data available from webscrapes of listings after Airbnb's new discrimination policy took effect in September 2016. Future work can explore whether the policy helped curb discrimination on the platform by measuring the extent of discrimination before and after the policy took effect. If user interface really does influence the extent of discrimination in Airbnb, then the prices of minority and white hosts should start converging. 
